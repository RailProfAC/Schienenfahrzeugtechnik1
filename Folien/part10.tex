% !TEX root = SFV-14033_SFT1.tex
\section{Fahrzeugkonstruktion}
\sectionpage

\subsection{Bauformen}
\subsectionpage
\frame{\frametitle{Konstruktionsprinzipien der Wagenk\"asten}
\framesubtitle{}
\begin{columns}[t] 
\begin{column}[T]{.5cm}
\end{column} 
     \begin{column}[T]{5.5cm} 
     \textbf{Differenzialbauweise}
     	\begin{itemize}
     		\item Fertigung aus Halbzeugen:
		\begin{itemize}
		\item Einzelteile einfach geformt
		\item Formgebung durch F\"ugen und Umformen
		\end{itemize}
     	\end{itemize}
	\textbf{Integralbauweise}
     	\begin{itemize}
     		\item Fertigung aus komplex geformten Elementen:
		\begin{itemize}
		\item z.B. Strangpressprofile
		\item Formgebung durch F\"ugen und Zerspanen
		\end{itemize}
     	\end{itemize}
	\textbf{Tragfunktion}
     	\begin{itemize}
     		\item Tragendes Untergestell
		\item Selbsttragender Wagenkasten
	\end{itemize}
     \end{column}
     	\begin{column}[T]{6cm} 
         	\begin{center}
            		\includegraphics[width=0.8\textwidth]{Integral}\rotatebox{90}{\tiny \color{gray} Quelle: Siemens Pressebild}\\
		Wagenkasten in Integralbauweise
        		\end{center}
     \end{column}
 \end{columns}
}


\subsection{Begrenzungen}
\subsectionpage

\frame{\frametitle{Lichtraumprofil streckenseitig}
\framesubtitle{}
\begin{columns}[t] 
     \begin{column}[T]{6cm} 
     	\begin{itemize}
		\item Streckenseitiges Lichtraumprofil muss ber\"ucksichtigen
		\begin{itemize}
		\item Beladungszust\"ande
		\item Dynamische Bewegungen:
		\begin{itemize}
		\item Ein-/Ausfedern
		\item Wanken
		\item Nicken
		\end{itemize}
		\item Bogenfahrt
		\item Kompatibilit\"at mit anderen Fahrzeugen
		\end{itemize}
     		\item Deutsches Regelprofil: G2
		\item Europ\"aisch: G1
		\item Betrieblich Ladema{\ss}\"uberschreitungen m\"oglich
     	\end{itemize}
     \end{column}
     	\begin{column}[T]{6cm} 
         	\begin{center}
			\only<1>{
            		\includegraphics[width=0.9\textwidth]{G2Strecke}\source{Quelle: Christian Lindecke} \\
			\small Lichtraumprofil G2 gem\"a{\ss} EBO}
			\only<2>{\includegraphics[width=0.9\textwidth]{TGCan}\\
			\small Kanadisches Lichtraumprofil}
        		\end{center}
     \end{column}
 \end{columns}
}

\frame{\frametitle{Fahrzeugbegrenzung: Querschnitt}
\framesubtitle{}
         	\begin{center}
            		\includegraphics[width=0.8\textwidth]{G1G2}\source{Quelle: Christian Lindecke}
        		\end{center}
     }


\offslide{Breiteneinschr\"ankung und Lichtraumbedarf}

\frame{\frametitle{Radsatzlasten und Meterlasten}
\framesubtitle{}
\begin{columns}[t] 
     \begin{column}[T]{6cm} 
     	\begin{itemize}
     		\item Beschr\"ankung der Radsatzlast:
		\begin{itemize}
		\item Gem\"a{\ss} Streckenkategorie
		\item Normativ, z.B. TSI Loc\&Pas (f\"ur HGV), EN 15528
		\end{itemize}
		\item Beschr\"ankung der Streckenlast
		\begin{itemize}
		\item z.B. f\"ur Br\"uckenbauwerke, Oberbau
		\end{itemize}
     	\end{itemize}
     \end{column}
     	\begin{column}[T]{6cm} 
         	\begin{center}
		\begin{tabular}{|c|c|c|}
		\hline
			Klasse & Radsatzlast & Meterlast \\ \hline
			A & 16 t & 5{,}0 t/m \\ \hline
			B1 & 18 t & 5{,}0 t/m \\ \hline
			B2 & 18 t & 6{,}4  t/m \\ \hline
			C2 & 20 t & 6{,}4  t/m \\ \hline
			C3 & 20 t & 7{,}2  t/m \\ \hline
			C4 & 20 t & 8{,}0  t/m \\ \hline
			D2 & 22{,}5 t & 6{,}4  t/m \\ \hline
			D3 & 22{,}5 t & 7{,}2  t/m \\ \hline
			D4 & 22{,}5 t & 8{,}0  t/m \\ \hline
			E4 & 25 t & 8{,}0  t/m \\ \hline
			E5 & 25 t & 8{,}8  t/m \\ \hline
		\end{tabular}
        		\end{center}
     \end{column}
 \end{columns}
}

\offslide{L\"angen- und Gewichtseinschr\"ankungen}

\subsection{Wagenkastenrohbau}
\subsectionpage
\frame{\frametitle{Anforderungen an den Wagenkasten \textit{car body}}
\framesubtitle{}
\begin{itemize}
\item Festigkeit (EN 12663):
\begin{itemize}
	\item Zug-/Druckkr\"afte im Zugverband
	\item Crash-Szenarien (EN 15227)
	\item Drucks\"o{\ss}e, Druckdichtigkeit
	\item Durchbiegung unter Beladung
	\item Schwingungen
\end{itemize}
\item Kunden-/ betriebliche Anforderungen
\begin{itemize}
	\item Lebensdauer
	\item Reparaturfreundlichkeit, Ersatzteilverf\"ugbarkeit
	\item Geringe Masse
	\item Design
	\item Entsorgung/Recycling
\end{itemize}
\item Normative/gesetzliche Anforderungen
\begin{itemize}
	\item Brandschutz (DIN 5510, EN 45545, ...)
	\item Material (EG 1907/2006 REACH)
	\item Crash und Festigkeit s.o.
\end{itemize}
\item Systemimmanente Anforderungen (Schwingungen, elastische Verformung,...)
\end{itemize}
}



\frame{\frametitle{Leichtbau der Wagenk\"asten}
\framesubtitle{}
\begin{itemize}
\item Alle Elemente an Aufnahme der Beanspruchungen beteiligen
\item Gut (leicht) ertragbar:
\begin{itemize}
	\item Zug- und Druckkr\"afte
\end{itemize}
\item Mit zus\"atzlichem Material ertragbar:
\begin{itemize}
	\item Torsions- und Biegemomente
\end{itemize}
\item H\"oherfeste Materialien werden z\"ogerlich angenommen
\begin{itemize}
	\item Bedenken bei Wartbarkeit und Lebensdauer
\end{itemize}
\end{itemize}
\begin{center}
\includegraphics[width=0.8\textwidth]{Class66}
\end{center}
}

\frame{\frametitle{Werkstoffe f\"ur Wagenk\"asten}
\framesubtitle{}
\begin{itemize}
\item Stahl:
\begin{itemize}
	\item Klassisch eingesetzt: Baust\"ahle S235, S355
	\item Ebenfalls anzutreffen: Edelst\"ahle, z.B. X5CrNi18-10
	\item Gut zu f\"ugen und umzuformen
	\item Dauerfestigkeit und elastisch/plastisches Verhalten gutm\"utig
\end{itemize}
\item Aluminium:
\begin{itemize}
	\item Geringere Dichte, geringerer E-Modul
	\item Dauerfestigkeitsgrenze wenig ausgepr\"agt
	\item Schwei{\ss}n\"ahte wenig erm\"udungsfest
	\item F\"ugeverfahren erfordern getrennte Behandlung von Stahl
\end{itemize}
\item Kunststoffe:
\begin{itemize}
	\item In der Regel faserverst\"arkt (GFK, CFK)	
	\item Erm\"oglichen Integralbauweise und Funktionsintegration
	\item Auch als Sandwichmaterialien
\end{itemize}
\item Waben und Schaummaterialien:
\begin{itemize}
	\item Eingesetzt im Deformationsbereich
\end{itemize}
\end{itemize}
}

\frame{\frametitle{Hauptbaugruppen des Rohbaus}
\framesubtitle{}
\begin{columns}[t] 
     \begin{column}[T]{6cm} 
     	\begin{itemize}
     		\item Untergestell
		\begin{itemize}
		\item (Mittel/Aussen-) Langtr\"ager
		\item Quertr\"ager
		\end{itemize}
		\item Seitenw\"ande
		\begin{itemize}
		\item Druckwechselbelastung
		\end{itemize}
		\item Dach
		\begin{itemize}
		\item Wasserablauf
		\end{itemize}
		\item Endw\"ande
		\begin{itemize}
		\item Schnittstellen, Crash
		\end{itemize}
		\item Kopfmodule
		\begin{itemize}
		\item Vorfertigung, Schnittstellen, Crash
		\end{itemize}
     	\end{itemize}
     \end{column}
     	\begin{column}[T]{6cm} 
         	\begin{center}
            		\includegraphics[width=0.8\textwidth]{FrontNose}
		\rotatebox{90}{\tiny \color{gray}{Quelle: Voith Pressebild}}
        		\end{center}
     \end{column}
 \end{columns}
}

\frame{\frametitle{Prozess Wagenkastenfertigung}
\framesubtitle{Bei allen Schritten zu beachten: Teils extremer Verzug durch W\"armeeinleitung}
\begin{columns}[t] 
     \begin{column}[T]{6cm} 
     	\begin{enumerate}
     		\item Einzelteilfertigung
		\begin{itemize}
		\item Schneiden, Schwei{\ss}nahtvorbereitung, Kanten, etc.
		\end{itemize}
		\item Baugruppenfertigung
		\begin{itemize}
		\item Schwei{\ss}en, evtl. Bearbeitung
		\item Hand- oder Roboterschwei{\ss}en je nach Naht
		\item Vermessung
		\end{itemize}
		\item Wagenkastenaufbau
		\begin{itemize}
		\item Vorsprengung bei statischer Durchbiegung
		\item Dichtigkeitspr\"ufung
		\end{itemize}
		\item Richten
		\item Sandstrahlen
     	\end{enumerate}
     \end{column}
     	\begin{column}[T]{6cm} 
         	\begin{center}
            		\includegraphics[width=0.8\textwidth]{Rohbau}\rotatebox{90}{\tiny \color{gray} Quelle: Siemens Pressebild}\\  
		Pr\"ufen der Aussenkontur     		
		\end{center}
     \end{column}
 \end{columns}
}
