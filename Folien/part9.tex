% !TEX root = SFV-14033_SFT1.tex
\section{Zugdynamik}
\frame{\sectionpage}
%\subsection{Einf\"uhrung Zugdynamik}
%\frame{\subsectionpage}
%
%\offslide{Einf\"uhrung Zugdynamik am Tafelbild}{Tafelbilder 1 - 5}

\subsection{Kuppelsto{\ss}, Crash}
\frame{\subsectionpage}

\offslide{Reversibler Energieverzehr: L\"osungen, Wirkungsgrade}

\frame{\frametitle{Crash: Anforderungen der EN15227 \citep{dinen15227}}
\framesubtitle{}
\begin{center}
\begin{tabular}{|c|l|c|c|c|c|}
\hline
\multirow{2}{*}{Szenario} & \multirow{2}{*}{Hindernis} &\multicolumn{4}{c|} {Kollisionsgeschwindigkeit $v_{c}$}  \\ \cline{3-6}
& & C I & C II & C III & C IV\\ \hline
1 & Identische Zugeinheit & 36 & 25 & 25 & 15 \\ \hline
\multirow{2}{*}{2} & G\"uterwagen 80 t & 36 & - & 25 & - \\ \cline{2-6}
 & 129 t Regionalzug & - & - & 10 & - \\ \hline
 \multirow{2}{*}{3} & Deformierbar 15 t & $v_{lc} - 50$ & - & 25 & - \\ \cline{2-6}
 & Starr 3 t & - & - & - & 25 \\ \hline
\end{tabular}
\end{center}
	\begin{itemize}
		\item Zus\"atzlich: Anforderungen an Bahnr\"aumer
		\item \"Uberlebensraum und maximale Verz\"ogerungen m\"ussen eingehalten werden
		\item Nachweis \"uber Komponententests und validierte Modelle m\"oglich
	\end{itemize}
}
\offslide{Umsetzung Anforderungen EN 15227}

\subsection{Kraftschluss, Schlupf}
\frame{\subsectionpage}

\offslide{Kr\"afte am Rad}{Tafelbild 11}

\offslide{Physikalische Kraftschlusstheorie}{Tafelbild 12}

\offslide{Kraftschluss-Schlupf-Gesetz}{Tafelbild 13}

\offslide{Radschlupf: weitere Einfl\"usse}{Tafelbild 14}

\subsection{Fahrwiderstand, Zugkraft, Zugbremsung}
\frame{\subsectionpage}
\offslide{Sammlung Fahrwiderst\"ande am Tafelbild}

\offslide{Zugkraftdiagramm am Tafelbild}

\frame{\frametitle{Modelle f\"ur Zugdynamik}
\framesubtitle{}
\begin{itemize}
\item Massenpunktmodell
\begin{itemize}
	\item z.B. Einzelfahrzeuge, \"Uberschlagsrechnungen
\end{itemize}
\item Homogenes starres Massenbandmodell
\begin{itemize}
	\item z.B. Reisez\"uge
\end{itemize}
\item Inhomogenes starres Massenbandmodell
\begin{itemize}
	\item z.B. lange G\"uterz\"uge
\end{itemize}
\item Elastisches homogenes Massenbandmodell
\begin{itemize}
	\item z.B. Triebz\"uge
\end{itemize}
\item Elastisches inhomogenes Massenbandmodell
\begin{itemize}
	\item Allgemeines Modell
\end{itemize}
\end{itemize}
}


\offslide{Neigungskraft am Tafelbild}

\offslide{Zugbremsung am Tafelbild}


