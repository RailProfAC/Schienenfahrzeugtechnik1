% !TEX root = SFT1-Skript.tex
% !TEX root = SFT1-Folien.tex
\section{Requirements Engineering}
\lehrtext{
In diesem Abschnitt der Veranstaltung sollt ihr lernen,
\begin{itemize}
	\item wo die Anforderungen an Schienenfahrzeuge herkommen,
	\item wie man systematisch mit ihnen arbeitet und
	\item welche Pr\"ufschritte euch (auch in Industrieprojekten) erwarten.
\end{itemize}
}

\frame{\frametitle{Warum Requirements Engineering (RE)?}
\lehrtext{Requirements Engineering befasst sich mit dem systematischen Erfassen, Umsetzen und Pr\"ufen von Anforderungen im Entwicklungsprozess. Durch striktes RE kann man Projektrisiken erheblich reduzieren und evtl. auch Kosten sparen.}
\begin{itemize}
\item Qualit\"at: Qualit\"at ist das Ma{\ss} der Erf\"ullung der Anforderungen an ein Produkt.
\item Kosten- und Termintreue
\item Einbindung der Stakeholder (Anspruchsteller)
\item Systematisierung der Beschaffung und des Engineerings
 
\end{itemize}
}

\frame{\frametitle{Key-Aspects of Requirements Engineering}
\lehrtext{
Was RE vom sonstigen Entwicklungsprozess unterscheidet ist vornehmlich, dass:
\begin{itemize}
	\item alle, die im Projekt Anspr\"uche haben (``Stakeholder'') k\"onnen sich einbringen,
	\item Pr\"ufungen werden als Reviews mit den Stakeholdern durchgef\"uhrt und
	\item jede Anforderung ist von Anforderung bis zum Nachweis der Erf\"ullung nachvollziehbar.
\end{itemize}
Beim dritten Punkt wird sichergestellt, dass eine geschlossene Kette von Verweisen von der Anforderung \"uber alle nachgelagerten Dokumente (Pflichtenheft, Konstruktionsunterlagen, Test-Spec...) f\"uhrt, die sicherstellt, dass diese Anforderung umgesetzt und erf\"ullt wurde. \href{https://de.wikipedia.org/w/index.php?title=Rückverfolgbarkeit_(Anforderungsmanagement)&oldid=197422472}{Traceability bei Wikipedia}
}
\begin{itemize}
\item Stakeholder Involvement
\item Technical Reviews
\item Traceability
\end{itemize}
}

%\offslide{Generisches Phasenmodell}{Modell einer beliebigen Phase eines Entwicklungsprozesses}

\frame{\frametitle{Generisches Phasenmodell}
\lehrtext{
Jede Phase eines Entwicklungsprozesses kann (sollte!) wie folgt abgebildet sein, damit die Qualit\"at des Entwicklungsschritts sichergestellt werden kann.

}
\begin{columns}[t] 
     \begin{column}[T]{5cm} 
     \textbf{F\"ur jede Phase festzulegen:}
     	\begin{itemize}
     		\item Purpose
		\item Inputs
		\item Entry Criteria
		\item Roles
		\item Verification steps
		\item Outputs
		\item Exit criteria
		\item Resources
		\item Management review activities
     	\end{itemize}
     \end{column}
     	\begin{column}[T]{7cm} 
         	\begin{center}
            		\includegraphics[width=1.0\textwidth]{Phase.png}
        		\end{center}
     \end{column}
 \end{columns}
}

%\offslide{V-Modell f\"ur Requirements Engineering}

\frame{\frametitle{V-Modell f\"ur Requirements Engineering}
\lehrtext{
Die einzelnen Phasen k\"onnen dem V-Modell entnommen werden (es empfiehlt sich sogar!). Bitte beachtet im Bild die Verbindungen zwischen absteigendem Pfad (Entwicklungsdokumente) und aufsteigendem Pfad (Pr\"ufdokumente): hier hat die Organisation die M\"oglichkeit, sich weiterzuentwickeln. \href{https://de.wikipedia.org/wiki/V-Modell_(Entwicklungsstandard)}{V-Modell bei Wikipedia}
}
         	\begin{center}
            		\includegraphics[width=0.9\textwidth]{VModel}\source{US DoT}
        		\end{center}
}

\frame{\frametitle{Requirements Analysis}
\lehrtext{Im ersten Schritt geht es um das Ermitteln der System Level Requirements, also der sehr hoch angesiedelten Anforderungen. Das h\"ort sich einfacher an, als es ist: viele Kunden wissen gar nicht, was sie brauchen. Daher liest man h\"aufig auch den Term ``Elicit Requirements (Anforderungen hervorlocken)'' daf\"ur.

Der erste Meilenstein fragt daher ab, ob beide Seiten die Anforderungen verstanden haben. Hier geht es auch um rudiment\"are Dinge wie den Ausgabestand des Lastenhefts.

}
\begin{columns}[t] 
     \begin{column}[T]{6cm} 
     \textbf{Leitfragen:}
     	\begin{itemize}
		\item What are the stakeholders?
     		\item What is the system to do?
		\item How well it is to do it?
		\item Under what conditions?
     	\end{itemize}
	\textbf{Typischer Meilenstein: Initial Design Review (IDR)}
     \end{column}
     	\begin{column}[T]{6cm} 
         	\begin{center}
            		\includegraphics[width=0.95\textwidth]{Phase}
        		\end{center}
     \end{column}
 \end{columns}
}

%\offslide{Erg\"anzen des generischen Phasenmodells}{}

\frame{\frametitle{System Specification}
\lehrtext{Im n\"acshten Level geht es um den Entwurf auf einer hohen Abstraktionsebene, also um die Architektur, L\"osungen und den Zuschnitt der Subsysteme.
}
\begin{columns}[t] 
     \begin{column}[T]{6cm} 
     \textbf{Leitfragen:}
     	\begin{itemize}
		\item Is the required system feasible?
     		\item What are system and subsystem borders?
		\item What are associated costs/lead times/risks?
		\item How can the risk be reduced?
		\item Which system integration steps are necessary?
     	\end{itemize}
	\textbf{Typischer Meilenstein: Preliminary Design Review (PDR)}
     \end{column}
     	\begin{column}[T]{6cm} 
         	\begin{center}
            		\includegraphics[width=0.95\textwidth]{Phase}
        		\end{center}
     \end{column}
 \end{columns}
}

%\offslide{Erg\"anzen des generischen Phasenmodells}%{Durchf\"uhrung in der \"Ubung}

\frame{\frametitle{Subsystem Design}
\lehrtext{
In diesem Schritt wird die Feinstruktur des Systems entwickelt (``Lower Level Design''), also die Architektur der Subsysteme, Detaill\"osungen und der Zuschnitt der Module.}
\begin{columns}[t] 
     \begin{column}[T]{6cm} 
     \textbf{Leitfragen:}
     	\begin{itemize}
		\item What are the subsystem requirements?
		\item Make or Buy?
		\item Which deliverables (e.g. documentation) are requested?
		\item What is the suitable subsystem structure?
     	\end{itemize}
	\textbf{Typischer Meilenstein: Critical Design Review (CDR)}
     \end{column}
     	\begin{column}[T]{6cm} 
         	\begin{center}
            		\includegraphics[width=0.95\textwidth]{Phase}
        		\end{center}
     \end{column}
 \end{columns}
}

%\offslide{Erg\"anzen des generischen Phasenmodells}%{Durchf\"uhrung in der \"Ubung}

\frame{\frametitle{Module Design}
\lehrtext{Jetzt geht es ins Eingemachte: im Module Design entwickelt man die ``Build to Specifications'', also Zeichnungen, Schemata usw., nach denen dann wirklich gefertigt bzw. beschafft wird. Dementsprechend muss das dann auch ordentlich \"uberpr\"uft werden - in der Regel ganz klassisch im Vier-Augen-Prinzip, also Pr\"ufung und Freigabe

}
\begin{columns}[t] 
     \begin{column}[T]{6cm} 
     \textbf{Leitfragen:}
     	\begin{itemize}
		\item How can the module be realised efficiently?
		\item What are critical characteristics of the module and its parts?
		\item Can service proven modules be used or adapted?
     	\end{itemize}
     \end{column}
     	\begin{column}[T]{6cm} 
         	\begin{center}
            		\includegraphics[width=0.95\textwidth]{Phase}
        		\end{center}
     \end{column}
 \end{columns}
}

%\offslide{Erg\"anzen des generischen Phasenmodells}

%ISO15288
