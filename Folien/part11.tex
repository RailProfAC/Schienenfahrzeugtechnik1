% !TEX root = SFV-14033_SFT1.tex
\section{Laufwerk (Fahrwerk)}
\frame{\sectionpage}

\frame{\frametitle{Grunds\"atzliche Anforderungen}
\framesubtitle{}
\begin{itemize}
\item \"Ubertragung und Ausgleich der Vertikallasten zwischen Rad und Schiene
\item Spurf\"uhrung des Fahrzeugs
\item \"Ubertragung und Begrenzung der dynamischen Kr\"afte, aufgrund von:
\begin{itemize}
		\item Gleislagefehlern
		\item B\"ogen
		\item Weichen
		\item Dynamik zwischen den Fahrzeugen
		\end{itemize} 
\item Wirksame D\"ampfung von angeregten Schwingungen
\item \"Ubertragung von Traktions- und Bremskr\"aften
\begin{center}
\includegraphics[width=0.6\textwidth]{Bogiediagram} \source{Quelle: Christophe Jacquet}
 \end{center}
\end{itemize}
}

\frame{\frametitle{Anatomie der Eisenbahndrehgestelle \textit{bogies}}
\framesubtitle{}
\begin{columns}[t] 
     \begin{column}[T]{5cm} 
     	\begin{itemize}
     		\item Rads\"atze \textit{wheelset}
		\item R\"ader \textit{wheels}
		\item Radsatzlager \textit{axlebox}
		\item Radsatzaufh\"angung \textit{suspension}
		\begin{itemize}
		\item Federn
		\item D\"ampfer
		\end{itemize}
		\item Begrenzungen und Anschl\"age
		\item Wagenkastenanbindung
		\item Drehgestellrahmen \textit{bogie frame}
     	\end{itemize}
     \end{column}
     	\begin{column}[T]{7cm} 
         	\begin{center}
            		\includegraphics[width=0.9\textwidth]{SchemaDG}\source{Quelle: Partim}
        		\end{center}
     \end{column}
 \end{columns}
}

\frame{\frametitle{Rads\"atze}
\framesubtitle{}
\begin{columns}[t] 
     \begin{column}[T]{6cm} 
     	\begin{itemize}
     		\item Unterscheidung:
		\begin{itemize}
		\item Innen-/Aussenlagerung
		\item Bremse
		\begin{itemize}
		\item Klotzbremse
		\item Radbremsscheibe
		\item Wellenbremsscheibe
		\end{itemize}
		\item Antriebe
		\begin{itemize}
		\item Symmetrisch
		\item Asymmetrisch
		\end{itemize}
		\end{itemize}
     	\end{itemize}
     \end{column}
     	\begin{column}[T]{6cm} 
         	\begin{center}
            		\includegraphics[width=0.8\textwidth]{Radsatz}\source{Quelle: Falk2}
        		\end{center}
     \end{column}
 \end{columns}
}

\frame{\frametitle{Radsatzlager}
\framesubtitle{}
\begin{columns}[t] 
     \begin{column}[T]{6cm} 
     	\begin{itemize}
	\item Heute \"uberwiegend W\"alzlager
     		\item Zylindrische Lager:
		\begin{itemize}
		\item Vorteile bei der \"Ubertragung von Radsatzlasten
		\item Wenig bis keine Querf\"uhrung
		\end{itemize}
		\item Konische Lager:
		\begin{itemize}
		\item Reduzierte ertragbare Radsatzlasten
		\item Sehr gute Querf\"uhrung
		\end{itemize}
     	\end{itemize}
     \end{column}
     	\begin{column}[T]{6cm} 
         	\begin{center}
            		\includegraphics[width=0.8\textwidth]{RadsatzGleitlager}\source{Quelle: Ketamin}
        		\end{center}
     \end{column}
 \end{columns}
}

\frame{\frametitle{R\"ader}
\framesubtitle{}
\textbf{Unterscheidung}
\begin{itemize}
\item Konstruktionsprinzip:
	\begin{itemize}
		\item Einteilig
		\item Bereift
	\end{itemize}
\item Querschnitt:
	\begin{itemize}
		\item Gerade
		\item S-f\"ormig
		\item Konisch
		\item Wellenform
	\end{itemize}	
\end{itemize}
}

\offslide{Begrenzungen und Anschl\"age}

\offslide{Federcharakteristika}

\frame{\frametitle{Verbindung Drehgestell - Wagenkasten}
\framesubtitle{}
\begin{columns}[t] 
     \begin{column}[T]{6cm} 
     	\begin{itemize}
     		 \item Drehpfanne
		\begin{itemize}
		\item Flach
		\item Kugelig
		\end{itemize}
		\item Drehbar um Drehzapfen
		\item Meist \"Ubertragung der L\"angskr\"afte
		\item Evtl. zus\"atzlich Abst\"utzung auf Gleitplatten
		\item Weitere Verbindungen:
		\begin{itemize}
		\item Wankst\"utze
		\item Schlingerd\"ampfer
		\end{itemize}
     	\end{itemize}
     \end{column}
     	\begin{column}[T]{6cm} 
         	\begin{center}
            		\includegraphics[width=0.8\textwidth]{Drehpfanne}\source{Quelle: Manuel Schneider}
        		\end{center}
     \end{column}
 \end{columns}
}



\frame{\frametitle{Radsatzaufh\"angung}
\framesubtitle{}
\begin{columns}[t] 
     \begin{column}[T]{6cm} 
     	\begin{itemize}
     		\item \"Ublich: zweistufige Federung
		\begin{itemize}
		\item Prim\"arstufe: 
		\begin{itemize}
		\item Radsatz gegen Drehgestellrahmen
		\item Beschleunigung bis 100 g
		\end{itemize}
		\item Sekund\"arstufe: 
		\begin{itemize}
		\item Drehgestellrahmen gegen Fahrzeug
		\item Hohe Anforderungen an D\"ampfung
		\end{itemize}
		\end{itemize}
		\item<3> Bei G\"uterwagen auch einstufige Federung 
     	\end{itemize}
     \end{column}
     	\begin{column}[T]{6cm} 
         	\begin{center}
			\only<1>{
            		\includegraphics[width=0.8\textwidth]{SAGThameslink}\source{Quelle: Siemens Pressebild}}
			\only<2>{
            		\includegraphics[width=0.8\textwidth]{WattLinkage}\source{Quelle: Cdang/Tennen-Gas}}
			\only<3>{
            		\includegraphics[width=0.8\textwidth]{RadsatzGleitlager}\source{Quelle: Ketamin}}
        		\end{center}
     \end{column}
 \end{columns}
}

\frame{\frametitle{Verbindung Drehgestell - Wagenkasten}
\framesubtitle{}
\begin{columns}[t] 
     \begin{column}[T]{4cm} 
     	\begin{itemize}
     		\item Drehpfanne
\begin{itemize}
		\item Flach
		\item Kugelf\"ormig
		\end{itemize}
		\item Hochanlenkung
		\item Tiefanlenkung
     	\end{itemize}
     \end{column}
     	\begin{column}[T]{8cm} 
         	\begin{center}
            		\includegraphics[width=0.9\textwidth]{Tiefanlenkung}\source{Quelle: Christian Lindecke}
        		\end{center}
     \end{column}
 \end{columns}
}


\frame{\frametitle{Drehgestellrahmen}
\framesubtitle{}
\begin{columns}[t] 
     \begin{column}[T]{6cm} 
     	\begin{itemize}
		\item Form:
		\begin{itemize}
     		\item H-Form
		\item O-Form
		\end{itemize}
		\item Herstellung:
		\begin{itemize}
		\item Schwei{\ss}en
		\item Gie{\ss}en
		\end{itemize}
     	\end{itemize}
     \end{column}
     	\begin{column}[T]{6cm} 
         	\begin{center}
            		\includegraphics[width=0.8\textwidth]{Y25cast}\source{Quelle: A1AA1A} \\ \vspace{.5cm}  \includegraphics[width=0.8\textwidth]{Y25weld} \source{Quelle: A1AA1A}

        		\end{center}
     \end{column}
 \end{columns}
}
%\frame{\frametitle{Drehgestell-Anbauten}
%\framesubtitle{}
%\begin{itemize}
%\item Sandung
%\item Spurkranzschmierung
%\item Antennen
%\end{itemize}
%}
