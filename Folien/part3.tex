% !TEX root = SFV-14033_SFT1.tex
\section{Grundlagen}
\frame{\sectionpage}

\frame{\frametitle{Vergleich Straße-Eisenbahn \footnote{Nach Minde, IVE Hannover, 2007}}
\begin{small}
\begin{tabular}{|l|l|}
  \hline                       
Haftwert Gummi-Fahrbahn ($\mu_H \approx 0,9$) & Haftwert Stahl-Stahl ($\mu_H \approx 0,15$)  \\ 
Kleine Massen ($m_{abb} \approx 0,8 t$) & Große Radsatzmassen ($m_{abb} \approx 20 t$)  \\ 
Kurze Bremswege & Lange Bremswege  \\ \hline
Max. zwei gekuppelte Fahrzeuge & Zugbildung  \\ \hline
Ausweichen möglich & Spurführung  \\ \hline
Fahren auf Sicht & Signalisierung  \\ \hline
Relativer Bremswegabstand & Absoluter Bremswegabstand  \\ \hline
Rückkopplung Bremspedal & Verzögerte Bremswirkung  \\
  \hline  
\end{tabular}
\end{small}
}

\frame{\frametitle{Verhältnisse}
\begin{columns}[t] % contents are top vertically aligned
     \begin{column}[T]{6cm} % each column can also be its own environment
     \begin{itemize}
     \item \glqq Gute Verhältnisse\grqq \ der Eisenbahn vergleichbar mit \glqq Ausnahmesituationen\grqq \ auf der Straße
     \end{itemize}
     \end{column}
     \begin{column}[T]{5cm} % alternative top-align that's better for graphics
         \begin{center}
            \includegraphics[width=0.8\textwidth]{Bremsweg2}
        \end{center}
     \end{column}
     \end{columns}
}

\frame{\frametitle{Merkmale der Schienenfahrzeuge (teilweise nach \cite{schindler})}
\framesubtitle{}
\begin{center}
\scriptsize
\begin{tabular}{|p{.8cm}|p{1.4cm}|p{1.4cm}|p{1.4cm}|p{1.4cm}|p{1.5cm}|p{1.5cm}|}
\hline
 & Stra{\ss}enbahn & Stadtbahn & U-Bahn & S-Bahn & XMU CR & XMU HST \\ \hline
Gleis & Im Stra{\ss}enraum & Gro{\ss}teil eigener Gleisk\"orper & Eigener Gleisk\"orper & Vollbahngleis & Vollbahngleis & Vollbahngleis, z.T. HGV-Trassen \\ \hline
Kurven-radius & $\geq 15\mathrm{m}$ & $\geq 25\mathrm{m}$ & $\geq 90\mathrm{m}$ & $\geq 180\mathrm{m}$ & $\geq 625\mathrm{m}$ & $\geq 1800\mathrm{m}$ \\ \hline
Zugsi-cherung & Sicht & Sicht/Signale & Signale & Signale & Signale & F\"uhrerstands-signalisierung \\ \hline
$d_{Hp}$ & $(300 \ldots$ $ 600)\mathrm{m}$ & $(500 \ldots $ $ 800)\mathrm{m}$ & $(500 \ldots $ $ 1000)\mathrm{m}$ & $(750 \ldots $ $ 3000)\mathrm{m}$ & $(3 \ldots $ $20)\mathrm{km}$ & $\gg 20\mathrm{km} $ \\ \hline
L\"ange Fzg  & $(20 \ldots 53)\mathrm{m}$ & $(25 \ldots  40)\mathrm{m}$ & $(25 \ldots  40)\mathrm{m}$ & $(25 \ldots  40)\mathrm{m}$ & $\approx 26 \mathrm{m}$ & $ \approx 26 (28)\mathrm{m} $ \\ \hline
L\"ange Zug  & $\leq 75 \mathrm{m}$ & $ \leq 75\mathrm{m}$ & $\leq 120\mathrm{m}$ & $\leq 300 \mathrm{m}$ & $\leq 400 \mathrm{m}$ & $ \leq 400 \mathrm{m} $ \\ \hline
$v_{max}$  & $70 \mathrm{km/h}$ & $ 100\mathrm{km/h}$ & $100 \mathrm{km/h}$ & $140 \mathrm{km/h}$ & $\leq 189 \mathrm{km/h}$ & $ \geq 190 \mathrm{km/h} $ \\ \hline
$a_{max}$ & $\leq 1.5 \frac{\mathrm{m}}{\mathrm{s}^2}$ & $\leq 1.5 \frac{\mathrm{m}}{\mathrm{s}^2}$ & $\leq 1.3 \frac{\mathrm{m}}{\mathrm{s}^2}$ & $\leq 1.15 \frac{\mathrm{m}}{\mathrm{s}^2}$ & $\leq 1.15 \frac{\mathrm{m}}{\mathrm{s}^2}$ & $\leq 1.15 \frac{\mathrm{m}}{\mathrm{s}^2}$ \\ \hline
$b_{max}$ & $\leq 3 \frac{\mathrm{m}}{\mathrm{s}^2}$ & $\leq 3 \frac{\mathrm{m}}{\mathrm{s}^2}$ & $\leq 1.3 \frac{\mathrm{m}}{\mathrm{s}^2}$ & $\leq 1.0 \frac{\mathrm{m}}{\mathrm{s}^2}$ & $\leq 1.5 \frac{\mathrm{m}}{\mathrm{s}^2}$ & $\leq 1.5 \frac{\mathrm{m}}{\mathrm{s}^2}$ \\ \hline
$F_{L, test}$ & $\leq 300 \mathrm{kN}$ & $\leq 600 \mathrm{kN}$ & $\leq 800 \mathrm{kN}$ & $\leq 1500 \mathrm{kN}$ & $\leq 1500 \mathrm{kN}$ & $\leq 1500 \mathrm{kN}$ \\ \hline
\end{tabular}
\end{center}
}

\frame{\frametitle{Netz: Unterscheidung und Bedeutung}
\framesubtitle{}
\begin{itemize}
\item Hauptbahnen (Vollbahnen):
	\begin{itemize}
		\item F\"ur nationalen und internationalen Durchgangsverkehr
		\item Zweigleisig, i.d.R. elektrifiziert
		\item Ausbau f\"ur hohe Fahrgeschwindigkeiten: Kurvenradien, Fahrbahn, Zugsicherung
	\end{itemize}
	\item Stra{\ss}enbahnen:
	\begin{itemize}
		\item Definition s.o.
		\end{itemize}
	\item Nebenbahnen:
	\begin{itemize}
		\item Alle anderen
		\end{itemize}
\end{itemize}
}

\frame{\frametitle{Transeurop\"aisches Eisenbahnnetz}
\framesubtitle{Wichtig f\"ur Fahrzeugklassifizerung nach TSI}
\begin{center}
\includegraphics[width=0.75\textwidth]{TEN}
\end{center}
}

\subsection{Regulative und normative Grundlagen}
\frame{\frametitle{Regulative R\"aume am Beispiel des Systems Kupplung}
\framesubtitle{}
\begin{center}
\includegraphics[width=1.0\textwidth]{CouplerWorld}\\
{\color{blue!55!green} UIC/TSI} {\color{red!80!black} AAR} {\color{red!20!brown!60!black} GOST} {\color{violet} Mix}
\end{center}
}

\subsection{Aussenanschriften}
\frame{\frametitle{Europ\"aische Fahrzeugnummer (Reisezugwagen)}
\framesubtitle{}
\begin{center}
\begin{displaymath}
\underbrace{\mathsf{D-DB}}_{\text{\tiny Land, Fahrzeughalter}}
 \underbrace{50}_{\text{\tiny Fzg.-typ}}
 \underbrace{80}_{\text{\tiny Land d. Registrierung}}
 \underbrace{26}_{\text{\tiny Gattung}}
  - \underbrace{81}_{\tiny v_{max,}\text{\tiny Energieversorgung}}
  \underbrace{111}_{\text{\tiny Nummer innerhalb BR}}
  -\underbrace{9}_{\text{\tiny Pr\"ufziffer}}
\end{displaymath}
erg\"anzt um Bauart-/Gattungsbezeichnung.
\end{center}
}

\frame{\frametitle{Europ\"aische Fahrzeugnummer (Triebfahrzeuge)}
\framesubtitle{}
\begin{center}
\begin{displaymath}
\underbrace{9}_{\text{\tiny Selbstfahrend}}
\underbrace{1}_{\text{\tiny Tfz-Typ}}
 \underbrace{80}_{\text{\tiny Land d. Registrierung}}
 \underbrace{6185}_{\text{\tiny Baureihe}}
 \underbrace{750}_{\text{\tiny Nummer innerhalb BR}}
  -\underbrace{9}_{\text{\tiny Pr\"ufziffer}}
  \underbrace{\mathsf{D-BASF}}_{\text{\tiny Land und Tfz-Halter}}
\end{displaymath}
\end{center}
Triebfahrzeugtyp:
\begin{enumerate}
	\item Verschiedene
	\item Elektrische Lokomotive
	\item Diesellokomotive
	\item Elektrischer Triebzug (HGV)
	\item Elektrischer Triebzug (au{\ss}er HGV)
	\item Dieseltriebzug
	\item Spezieller Beiwagen
	\item Elektrische Rangierlokomotive
	\item Diesel-Rangierlokomotvie
	\item Sonderfahrzeug
\end{enumerate}
}


\frame{\frametitle{(Internationale) Verwendbarkeit}
\framesubtitle{}
\begin{itemize}
\item Internationale Verwendbarkeit laut RIC Raster, ggf. einzelne L\"ander
\item Angaben: $v_{max}$, L\"ander gem\"a{\ss} Vereinbarung, F\"ahrentauglichkeit, Stromarten zentrale Energieversorgung, weitere Ausr\"ustung
\end{itemize}
\begin{center}
            	\includegraphics[width=0.7\textwidth]{RICRaster}\\ \vspace{.2cm}	
		\includegraphics[width=0.7\textwidth]{RIC} \rotatebox{90}{{\tiny \color{gray} Quelle: Tobias K\"ohler}} 
		
        		\end{center}
}


