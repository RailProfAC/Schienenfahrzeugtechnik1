% !TEX root = SFT1-Skript.tex
\section{G\"uterwagen}
\frame{\sectionpage}


\frame{\frametitle{Zahlen zum G\"uterverkehr}
\framesubtitle{}
\begin{center}
\begin{tikzpicture}[scale = 1]
		\begin{axis}[width = 10 cm, ybar =2pt, bar width = 3pt,
    		xlabel={Jahr},
%    		ylabel={\color{blue!80!black}Mrd. tkm},
		%xtick=data,
		xtick={1997,2000,2005,2010,2015},
		x tick label style={rotate=45,anchor=east},
		ymin = 0, ymax = 125,
		legend style={at={(0.5,-0.30)},
        		anchor=north,legend columns=-1},
		/pgf/number format/.cd,
        		use comma,
        		1000 sep={},]
\addplot table {SGVMenge.dat};
\addplot table {SGVWagen.dat};
%\addplot table {data_d3.dat};
%\addplot table {data_d4.dat};
\legend{Mrd. tkm, 1000 G\"uterwagen}
\end{axis}
\end{tikzpicture}
\end{center}
}



\frame{\frametitle{Einf\"uhrung}
\framesubtitle{}
\begin{columns}[t] 
     \begin{column}[T]{5cm} 
     	\begin{itemize}
     		\item Gr\"o{\ss}te Gruppe an Fahrzeugen
		\item Universalwagen
		\begin{itemize}
		\item Standardisierte Verkehre
		\item z.B. Flachwagen
		\end{itemize}
		\item Sonderbauart 
		\begin{itemize}
		\item Bestimmte Verkehre
		\item z.B. Containertragwagen, Pkw-Transportwagen
		\end{itemize}     	
		\item H\"aufig im Privatbesitz
		\item Anspruchsvoll trotz einfacher Technik
		\end{itemize}
     \end{column}
     	\begin{column}[T]{7cm} 
         	%\begin{center}
		Regelbauart:
	     	\begin{tabular}{|l|c|}
		\hline 
		E & offene Wagen \\ \hline
		G & gedeckte Wagen \\ \hline
		K & Flachwagen (2 RS) \\ \hline
		O & gemischte Offen-Flachwagen \\ \hline
		R & Drehgestell-Flachwagen \\ \hline
		\end{tabular} \vspace{1mm}
		Sonderbauart:
		\begin{tabular}{|l|c|}
		\hline 
		F & offene Wagen \\ \hline
		H & gedeckte Wagen \\ \hline
		I & K\"uhlwagen \\ \hline
		L & Flachwagen mit unabh\"angigen RS \\ \hline
		S & Drehgestell-Flachwagen \\ \hline
		T & Wagen mit \"ofnungsf\"ahigem Dach \\ \hline
		U & Sonderwagen \\ \hline
		Z & Kesselwagen \\ \hline
		\end{tabular}
            	%\end{center}
     \end{column}
 \end{columns}
}

\offslide{Sammeln von Anforderungen}

\frame[allowframebreaks]{\frametitle{Anforderungen gem\"a{\ss} WAG TSI}
\framesubtitle{}
\begin{itemize}
\item Festigkeit gem\"a{\ss} EN12663-2
\begin{itemize}
	\item Zwei Kategorien: F-I: Allgemein, F-II: nicht ablaufen/absto{\ss}en
	\item L\"angsdruckkraft: F-I: 2000 kN, F-II: 1200 kN
	\item Zugkraft: 1000/1500 kN (je nach Anschlag)
\end{itemize}
\item Integrit\"at: bewegliche TEile sind gegen Positions\"anderungen gesichert
\item Begrenzungslinie abh\"angig vom Zielprofil
\item Radsatzlast gem\"a{\ss} EN 15228
\item Kompatibilit\"at mit Gleisfreimeldeanlagen
\begin{itemize}
	\item Gleisstromkreise
	\item Achsz\"ahler
	\item Kabelschleifen
\end{itemize}
\item Zustands\"uberwachung der Radsatzlager
\begin{itemize}
	\item Fahrzeugseitig 
	\item Streckenseitig (gem\"a{\ss} EN15437-1:2009)
\end{itemize}
\item Laufsicherheit
\begin{itemize}
	\item Sicherheit gegen Entgleisen unter Gleisverwindung
	\item Dynamisches Verhalten gem. EN14363 oder mittels validiertem Modell
\end{itemize}
\item Laufwerk
\begin{itemize}
	\item Festigkeit gem\"a{\ss} EN13749
	\item Forderungen an Rads\"aze und R\"ader gem\"a{\ss} WAG TSI
\end{itemize}
\item Bremse
\begin{itemize}
	\item Sicherheitsbetrachtung gem\"a{\ss} Common Safety Methods (CSM, (EG) Nr. 352/2009)
	\begin{itemize}
		\item Ausfall einer Einheit bei Mehrfachfehler
		\item Ausfall mehrerer Einheiten bei Einfachfehler
	\end{itemize}
	\item Bremsleistung
	\begin{itemize}
		\item Durch Berechnung gem\"a{\ss} EN14531-6
		\item Durch Versuch gem\"a{\ss} UIC 544-1
	\end{itemize}
	\item Feststellbremse
	\begin{itemize}
		\item Zustandsanzeige
	\end{itemize}
	\item W\"armekapazit\"at
	\begin{itemize}
		\item Dauerbremsung mit 45 kW (70 km/h, 40 km, i = 2,1\%)
	\end{itemize}
	\item Gleitschutz f\"ur Scheibenbremsen oder Klotzbremse mit $\mu_{m} > 0{,}12$
\end{itemize}
\item Umgebungsbedingungen
\begin{itemize}
	\item T1: -25 \si{\degreeCelsius} bis +40 \si{\degreeCelsius}
	\item T2: -40 \si{\degreeCelsius} bis +35 \si{\degreeCelsius}
	\item T3: -25 \si{\degreeCelsius} bis +45 \si{\degreeCelsius}
	\item Schnee, Eis und Hagel gem\"a{\ss} EN50125-1
\end{itemize}
\item Brandschutz
\begin{itemize}
	\item Abschirmung potenzieller Brandquelle von der Ladung
	\item Anforderungen an Materialien, Kabel und Fl\"ussigkeiten
\end{itemize}
\item Dokumentation
\begin{itemize}
	\item Betriebsunterlagen
	\item Instandhaltungsvorschriften
\end{itemize}
\end{itemize}
}

\frame{\frametitle{EG-Konformit\"at nach TSI WAG}
\framesubtitle{}
\begin{itemize}
\item F\"ur einige Elemente (Interoperabilit\"atskomponenten) wird von einer EG-Konformit\"at ausgegangen:
\begin{itemize}
	\item Einachsige Laufwerke: Doppelschakenaufh\"angung, Niesky 2, S 2000
	\item Drehgestelle mit zwei Rads\"atzen: Y25-Familie, zweiachsiges Lenkdrehgestell
	\item Dreiachsige Drehgestelle mit Schakenaufh\"angung
\end{itemize}
\item Auch f\"ur gewisse Materialien in Bezug auf Entflammbarkeit sowie Brandschutzw\"ande
\end{itemize}
}

