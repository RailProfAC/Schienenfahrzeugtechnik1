% !TEX root = SFV-14033_SFT1.tex
\section{System\"uberblick}
\frame{\sectionpage}

\frame{\frametitle{Definition Schienenfahrzeug}
\framesubtitle{Definition gem\"a{\ss} DIN 25003}
\begin{definition}[Schienenfahrzeuge \textit{rail vehicles}]
Spurgebundene Fahrzeuge, die mit Spurkranz versehenen R\"adern auf Gleisen, die aus Schienen einer bestimmten gleichbleibenden Spurweite gebildet sind, gef\"uhrt und getragen werden.
\end{definition}
Unterscheidung:
\begin{itemize}
	\item Eisenbahnfahrzeuge (gem\"a{\ss} AEG und EBO/ESBO)
	\item Strassenbahnen (gem\"a{\ss} PBefG und BOSTRAB)
	\item Nicht \"offentliche Bahnen (z.B. Werksbahnen) (gem\"a{\ss} BOA und EBOA)
\end{itemize}
}

% Din 25003
%Einführung in die Verkehrstechnik
%Zahlen und Fakten zum Verkehr
%Abgrenzung zur Fördertechnik
%Grundfunktionen des Schienenfahrzeugs
\subsection{Systematik}
\subsection{Systematik der Eisenbahnen}
\frame{\frametitle{Systematik des Eisenbahnverkehrs}
\framesubtitle{}
\begin{center}
\begin{tikzpicture}[thick]
		%\tiny
  \node[draw,rectangle, anchor = west] (a) {Eisenbahnen};
  \node[element, below of = a, left of = a, anchor = east] (b) {Personenverkehr};
  \node[element, below of = a, right of = a, anchor = west] (c) {G\"uterverkehr};
 
   \node[element, below of = b, left of = b, anchor = east] (e) {Fernverkehr}; 
  \node[element, below of= e](f) {Regionalverkehr};
  \node[element, below of=f] (g) {Intercity-Verkehr};
  \node[element, below of=g] (h) {Hochgeschwin-digkeitsverkehr};
  
  
  
 \node[element, below of = b, right of = b, anchor = west] (j) {Nahverkehr};
  \node[element, below of= j](k) {Stra{\ss}enbahn \\ \textit{Tram}};
  \node[element, below of=k] (l) {U-Bahn \\ \textit{Metro}};

%\node[element, below of=k] (l) {Triebzug \textit{Multiple Unit}};
 

    \draw[-] (a) -| (b);
     \draw[-] (a) -| (c);
    
    \draw[-] (b) -| (e);
    \draw[-] (b) -| (j); 
    
    \draw[-] (e) -- (f);
    \draw[-] (f) -- (g);
    \draw[-] (g) -- (h);
    
    \draw[-] (j) -- (k);
    \draw[-] (k) -- (l);
    
\end{tikzpicture}
\end{center}
}

\subsubsection{Systematik der Eisenbahnfahrzeuge}
\frame{\frametitle{Systematik der Eisenbahnfahrzeuge}
\framesubtitle{}
\begin{center}
\tikzstyle{element} = [draw,rectangle, align = center, text width = 2.5 cm, node distance = 1.25 cm]
\begin{tikzpicture}[thick]
		%\tiny
  \node[draw,rectangle] (a) {Eisenbahnfahrzeuge};
  \node[right of = a] (a1) {};
  \node[draw,rectangle, below of=a, minimum width = 4 cm, left of = a, anchor = east] (b) {Regelfahrzeuge};
  \node[left of = b] (b1) {};
  \node[right of = b] (b2) {};
  \node[draw,rectangle,below of=a, right of = a, anchor = west] (c) {Nebenfahrzeuge}; 
   \node[element, below of=b, left of = b, anchor = east] (d) {Wagen};
    \node[element, below of=d] (f) {G\"uterwagen \\ \textit{wagons}};
\node[element, below of=f] (g) {Reisezugwagen \textit{coaches}};
\node[element, below of=g] (h) {Sonstige \\ \textit{others}};

  \node[element, below of = b, right of = b, anchor = west] (e) {Triebfahrzeuge}; 
  \node[element, below of=e] (j) {Lokomotiven \textit{locomotives}};
\node[element, below of=j] (k) {Triebwagen \textit{railcar}};
\node[element, below of=k] (l) {Triebzug \textit{Multiple Unit}};
 

    \draw[-] (a) |- (b);
    \draw[-] (a) |- (c);
    \draw[-] (b) |- (d);
    \draw[-] (b) |- (e);
    
    \draw[-] (d) -- (f);
    \draw[-] (f)  -- (g);
    \draw[-] (g) -- (h);
    
    \draw[-] (e) -- (j);
    \draw[-] (j)  -- (k);
    \draw[-] (k) -- (l);
  
\end{tikzpicture}
\end{center}
}

%\offslide{Fahrzeuggattungen}{Tafelbild 6}



