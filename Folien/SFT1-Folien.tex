\documentclass[slidestop,compress,mathserif, aspectratio = 169]{beamer}
%\usecolortheme{seagull} % Beamer color theme
\usetheme{metropolis}
\usepackage{etex}
%\usepackage[bars]{beamerthemetree} % Beamer theme v 2.2
%\usetheme{Madrid} % Beamer theme v 3.0
%\usecolortheme{seagull} % Beamer color theme
\usepackage{bibentry}
\usepackage[round]{natbib}
\usepackage{graphicx}
\usepackage{epigraph}
\usepackage{makeidx}
\usepackage{svg}
\makeindex
\usepackage[]{amsmath}
\usepackage{pgfcalendar}
\usepackage[]{amssymb}
\usepackage{color}
\usepackage{pict2e}
\usepackage{algorithm2e}
\usepackage[ngerman]{babel}
\usepackage{tikz}
\usetikzlibrary{decorations}
\usetikzlibrary{calc}
\usetikzlibrary{scopes}
\usetikzlibrary{arrows.meta}
\usetikzlibrary{mindmap}
\usepackage{multirow}
\usepackage[utf8]{inputenc}
\usepackage{multimedia}
\usepackage{xcolor}
\usepackage{keyval}
\usepgfmodule{shapes}
\usepackage{sfmath}
\usepackage{pgfplots}
\usepackage{sansmath}
\usepackage{siunitx}
\usepackage{multirow}

\definecolor{mint}{RGB}{32,178,170}
\graphicspath{{../../../figures/}}

\makeatletter
\pgfdeclareshape{loco}{
\inheritsavedanchors[from=rectangle] % this is nearly a rectangle
\inheritanchorborder[from=rectangle]
\inheritanchor[from=rectangle]{center}
\inheritanchor[from=rectangle]{north}
\inheritanchor[from=rectangle]{base}
\inheritanchor[from=rectangle]{south}
\inheritanchor[from=rectangle]{west}
\inheritanchor[from=rectangle]{east}
\backgroundpath{
\southwest \pgf@xa=\pgf@x \pgf@ya=\pgf@y
\northeast \pgf@xb=\pgf@x \pgf@yb=\pgf@y
\pgf@ya=\pgf@ya \advance\pgf@ya by 4pt
\pgf@xc=\pgf@xb \advance\pgf@xc by-3pt % this should be a parameter
\pgf@yc=\pgf@yb \advance\pgf@yc by-7pt
%\pgf@xd=\pgf@xa \advance\pgf@xc by+3pt % this should be a parameter
% construct main path
\pgfpathmoveto{\pgfpoint{\pgf@xa}{\pgf@ya}}
\pgfpathlineto{\pgfpoint{\pgf@xa}{\pgf@yc}}
\pgfpathlineto{\pgfpoint{\pgf@xa+3pt}{\pgf@yb}}
\pgfpathlineto{\pgfpoint{\pgf@xc}{\pgf@yb}}
\pgfpathlineto{\pgfpoint{\pgf@xb}{\pgf@yc}}
\pgfpathlineto{\pgfpoint{\pgf@xb}{\pgf@ya}}
\pgfpathclose
\pgfpathmoveto{\pgfpoint{\pgf@xa}{\pgf@ya-1pt}}
\pgfpathlineto{\pgfpoint{\pgf@xa+10pt}{\pgf@ya-1pt}}
\pgfpathmoveto{\pgfpoint{\pgf@xb}{\pgf@ya-1pt}}
\pgfpathlineto{\pgfpoint{\pgf@xb-10pt}{\pgf@ya-1pt}}
\pgfpathcircle{\pgfpoint{\pgf@xb-7pt}{\pgf@ya-2.5pt}}{1.5pt}
\pgfpathcircle{\pgfpoint{\pgf@xb-10pt}{\pgf@ya-2.5pt}}{1.5pt}
%\pgfpathcircle{\pgfpoint{\pgf@xa+7pt}{\pgf@ya-2.5pt}}{1.5pt}
%\pgfpathcircle{\pgfpoint{\pgf@xa+3pt}{\pgf@ya-2.5pt}}{1.5pt}
}
}
\makeatother


%\mode<handout>{
%\usepackage{pgfpages}
%\pgfpagesuselayout{9 on 1}[a4paper,border shrink=5mm]
%}

%\AtBeginSection{\frame{\sectionpage}}
\newtranslation[to=ngerman]{Section}{Teil}
\newtranslation[to=ngerman]{Subsection}{Abschnitt}

\pgfplotsset{width=5cm, height = 5 cm,
	tick label style = {font=\sansmath\sffamily},
  every axis label = {font=\sansmath\sffamily},
  legend style = {font=\sansmath\sffamily},
  label style = {font=\sansmath\sffamily}}
%\definecolor{RYB1}{RGB}{240,249,232}
%\definecolor{RYB2}{RGB}{186,228,188}
%\definecolor{RYB3}{RGB}{123,204,196}
%\definecolor{RYB4}{RGB}{67,162,202}
%\definecolor{RYB5}{RGB}{18,104,172}
%
%\definecolor{RYB11}{RGB}{141,211,199}
%\definecolor{RYB12}{RGB}{255,255,179}
%\definecolor{RYB13}{RGB}{190,186,218}
%\definecolor{RYB14}{RGB}{251,128,114}
%\definecolor{RYB15}{RGB}{128,177,211}
%\definecolor{RYB16}{RGB}{253,180,98}
%\definecolor{RYB17}{RGB}{179,222,105}
%\definecolor{RYB18}{RGB}{252,205,229}
%\definecolor{RYB19}{RGB}{217,217,217}
%\definecolor{RYB20}{RGB}{188,128,189}
%\definecolor{RYB21}{RGB}{204,235,197}
%\definecolor{RYB22}{RGB}{255,237,111}
%
%
\begin{document}
\tikzstyle{element} = [draw,rectangle, align = center, text width = 2.5 cm, node distance = 1.25 cm]

\newcommand\mydots{\makebox[1em][c]{.\hfil.\hfil.}}

%% Formalia
\newcommand{\revision}{Rev. 02}
\newcommand{\docnum}{SFV-14033}

\newcommand{
\offslide}[2]{
\begin{frame}
\frametitle{\includegraphics[scale=0.05] {Offwhite} \hspace{.5mm} #1}
%\framesubtitle{#2}
\begin{tikzpicture}[x=1cm, y=1cm, semitransparent]
\draw[step=5mm, line width=0.2mm, black!40!white] (0,0) grid (\textwidth,\textheight-1cm);
\node[anchor = south east] at (\textwidth,0) {#2};

\end{tikzpicture}
\end{frame}
}

\newcommand{\source}[1]{\rotatebox{90}{\tiny \color{gray} #1}}


\title{84111 Schienenfahrzeugtechnik I}
\subtitle{}
\author[R. Pfaff]{Prof. Dr. Raphael Pfaff}
\institute[Dokument \docnum, \revision]{Fachhochschule Aachen}
\logo{\put(-18, 120){\includegraphics[width=1.4cm]{logoR}}}
{
\usebackgroundtemplate{
{\transparent{.3}\includegraphics[width= \paperwidth]{EmmaCastlepale}}
%\begin{tikzpicture}
%\node[opacity=.1] {\includegraphics[width=\paperwidth]{Innotrans2018.jpg}};
%\end{tikzpicture}
}

\begin{frame} % Cover slide
\titlepage
\end{frame}
\usebackgroundtemplate{}
% Pr\"aliminarien
% !TEX root = SFV-14033_SFT1.tex
\section{Pr\"aliminarien}

%\begin{frame}[plain]
%\begin{center}
%\vspace{2cm}
%\Huge Warum seid Ihr heute hier?\\ \vspace{.5cm}
%\pause \Large Die Zahlentheorie ist nützlich, weil man mit ihr promovieren kann.\\
%Edmund Landau
%\end{center}
%\end{frame}


\begin{frame} % Cover slide
\titlepage
\end{frame}

\frame{\sectionpage}

\frame{\frametitle{Prof. Dr. Raphael Pfaff}
\framesubtitle{Lehr- und Forschungsgebiet Schienenfahrzeugtechnik}
\begin{columns}[t] 
     \begin{column}[T]{7cm} 
     	\begin{itemize}
		\item[] \includegraphics[width=0.4cm]{Email} \hspace{.1cm} pfaff@fh-aachen.de
		\item[] \includegraphics[width=0.4cm]{Twitter} \hspace{.1cm} @RailProfAC
		\item[] \includegraphics[width=0.4cm]{Wordpress} \hspace{.1cm} www.raphaelpfaff.net
		\item[] Prezume: \texttt{\url{http://goo.gl/iq6lhh}}
		\vspace{1cm}
		\item Raum 02305
		\item Sprechstunde nach Vereinbarung
     	\end{itemize}
	
     \end{column}
     	\begin{column}[T]{5cm} 
         	\begin{center}
            		\includegraphics[width=0.8\textwidth]{Profilklein}
        		\end{center}
     \end{column}
 \end{columns}
}

\frame{\frametitle{Vorstellungsrunde}
\framesubtitle{}
\begin{itemize}
\item Wer bist Du?
\item Was erwartest Du von SFT1?
\item Was kann ich tun, damit SFT1 Dein Traummodul wird?
\item Was muss ich tun, damit Du SFT1 hasst?
\end{itemize}
}


\frame{\frametitle{Anforderungen ``First Cycle'' - Bachelor}
\framesubtitle{Anforderungen gem\"a{\ss} Dublin Descriptors}
\begin{columns}[t] 
     \begin{column}[T]{7cm} 
     	\begin{itemize}
     		\item Knowledge and understanding in a field of study
		\begin{itemize}
		\item Typically supported by textbooks
		\item Some aspects informed by knowledge on the forefront of the field of study
		\end{itemize}
		\item Apply knowledge and understanding indicating a professional approach
		\item Gather and interpret data to inform judgement
		\item Communicate information, ideas, problems and solutions
		\item Learning skills to undertake further study with high degree of autonomy
     	\end{itemize}
     \end{column}
     	\begin{column}[T]{5cm} 
         	\begin{center}
	\vspace{1cm}
            		\includegraphics[width=0.8\textwidth]{GraduationHat}
        		\end{center}
     \end{column}
 \end{columns}
}

\frame{\frametitle{Anforderungen ``Niveau 6'' - Bachelor}
\framesubtitle{Anforderungen gem\"a{\ss} Deutschem Qualifizierungsrahmen}
\begin{columns}[t] 
     \begin{column}[T]{7cm} 
     	\begin{itemize}
     		\item Breites und integriertes Wissen
		\begin{itemize}
		\item Wissenschaftliche Grundlagen
		\item Praktische Anwendungen 
		\end{itemize}
		\item Breites Spektrum an Methoden
		\begin{itemize}
		\item Neue L\"osungen erarbeiten und bewerten
		\end{itemize}
		\item Verantwortlich in Expertenteams arbeiten oder leiten
		\item Ziele f\"ur Lern- und Arbeitsprozesse definieren, reflektieren und bewerten
		\item Lern- und Arbeitsprozesse eigenst\"andig und nachhaltig gestalten
     	\end{itemize}
     \end{column}
     	\begin{column}[T]{5cm} 
         	\begin{center}
	\vspace{1cm}
            		\includegraphics[width=0.8\textwidth]{GraduationHat}
        		\end{center}
     \end{column}
 \end{columns}
}


\frame{\frametitle{Anforderungen BEng Schienenfahrzeugtechnik}
\framesubtitle{Was zeichnet einen Bachelor der Schienenfahrzeugtechnik aus?}
\begin{columns}[t] 
     \begin{column}[T]{6cm} 
     	\begin{itemize}
		\item Wissenschaftliches Arbeiten
		\begin{itemize}
		\item Nutzung Prim\"arliteratur und Normen
		\item Erstellung Seminararbeiten
		\end{itemize}
     		\item Selbstlernkompetenz
		\begin{itemize}
		\item Beispiel: Nutzung Lehrbuch statt Skript
		\end{itemize}
		\item Verfassung wissenschaftlicher und technischer Texte
		\item Fachvortrag zu Seminararbeit
     	\end{itemize}
     \end{column}
     	\begin{column}[T]{6cm} 
         	\begin{center}
	\vspace{1cm}
            		\includegraphics[width=0.8\textwidth]{GraduationHat}
        		\end{center}
     \end{column}
 \end{columns}
}

\frame{\frametitle{Wie schaffen? - Growth Mindset!}
\framesubtitle{Theorie der Psychologin Carol Dweck, Harvard \cite{mindset}}
\begin{columns}[t] 
     \begin{column}[T]{6cm} 
     \textbf{Fixed Mindset:} 
     	\begin{itemize}
     		\item F\"ahigkeiten,
		\item Intelligenz und
		\item Talent 
     	\end{itemize}
	sind feste Pers\"onlichkeitsmerkmale und nicht \"anderbar. Daher das Bestreben, nicht dumm zu wirken.
     \end{column}
     	\begin{column}[T]{6cm} 
	\textbf{Growth Mindset:} 
         	\begin{itemize}
     		\item F\"ahigkeiten,
		\item Intelligenz und
		\item Talent 
     		\end{itemize}
		k\"onnen durch Anstrengung, gute Lehre und Hartn\"ackigkeit entwickelt werden. Nicht jeder ist gleich, aber jeder kann sich weiterentwickeln.
     \end{column}
 \end{columns}
}

\frame{\frametitle{Entwicklung im Growth Mindset}
\framesubtitle{}
\begin{center}
\begin{tikzpicture}[scale = .9, minimum height = 20]
\node (fixed)[minimum width = 70, draw=red!50!black, fill = red!20, rounded corners] at ( .6,8.75) {\textcolor{red!50!black}{Fixed mindset}};
\node (growth)[minimum width = 70, draw=green!50!black, fill = green!20, rounded corners]  at ( 4.3,8.75){\textcolor{green!50!black}{Growth mindset}};

\node (challenges)[text width = 320, draw=blue!30, fill = blue!20, align = left, minimum height = 24] at ( 0,7.5) {\textcolor{blue!40!black}{Herausforderungen}};
\node (obstacles)[text width = 320, draw=blue!30, fill = blue!20, align = left, minimum height = 24] at ( 0,6.25) {\textcolor{blue!40!black}{Hindernisse}};
\node (effort)[text width = 320, draw=blue!30, fill = blue!20, align = left, minimum height = 24] at ( 0,5) {\textcolor{blue!40!black}{Anstrengung}};
\node (criticism)[text width = 320, draw=blue!30, fill = blue!20, align = left, minimum height = 24] at ( 0,3.75) {\textcolor{blue!40!black}{Kritik}};
\node (success)[text width = 320, draw=blue!30, fill = blue!20, align = left, minimum height = 24] at ( 0,2.5) {\textcolor{blue!40!black}{Erfolg anderer}};

\begin{scope}[draw = red!50!black]
\uncover<2-7>{\node (avoid)[minimum width = 70, draw, fill = red!20, rounded corners] at ( .6,7.5) {\textcolor{red!50!black}{Vermeiden}};
\path[draw, draw=red!50, line width = 5pt] (fixed) -- (avoid);};
\uncover<3-7>{\node (giveup)[minimum width = 70, draw, fill = red!20, rounded corners] at ( .6,6.25) {\textcolor{red!50!black}{Aufgeben}};
\path[draw, draw=red!50, line width = 5pt] (avoid) -- (giveup);};
\uncover<4-7>{\node (useless)[minimum width = 70, draw, fill = red!20, rounded corners] at ( .6,5) {\textcolor{red!50!black}{Nutzlos}};
\path[draw, draw=red!50, line width = 5pt] (giveup) -- (useless);};
\uncover<5-7>{\node (ignore)[minimum width = 70, draw, fill = red!20, rounded corners] at ( .6,3.75) {\textcolor{red!50!black}{Ignorieren}};
\path[draw, draw=red!50, line width = 5pt] (useless) -- (ignore);};
\uncover<6-7>{\node (threat)[minimum width = 70, draw, fill = red!20, rounded corners] at ( .6,2.5) {\textcolor{red!50!black}{Bedrohung}};
\path[draw, draw=red!50, line width = 5pt] (ignore) -- (threat);};
\end{scope}

\begin{scope}[draw = green!50!black]
\uncover<2-7>{\node (embrace)[draw, minimum width = 70, fill = green!20, rounded corners] at ( 4.3,7.5) {\textcolor{green!50!black}{Nutzen}};
\path[draw,  line width = 5pt] (growth) -- (embrace);};
\uncover<3-7>{\node (persist)[draw, minimum width = 70,  fill = green!20, rounded corners] at (4.3,6.25) {\textcolor{green!50!black}{Weitermachen}};
\path[draw,  line width = 10pt] (embrace) -- (persist);};
\uncover<4-7>{\node (valuable)[draw, minimum width = 70,  fill = green!20, rounded corners] at ( 4.3,5) {\textcolor{green!50!black}{Wertvoll}};
\path[draw,  line width = 15pt] (persist) -- (valuable);};
\uncover<5-7>{\node (learn)[draw, minimum width = 70,  fill = green!20, rounded corners] at ( 4.3,3.75) {\textcolor{green!50!black}{Lernen}};
\path[draw,  line width = 20pt] (valuable) -- (learn);};
\uncover<6-7>{\node (inspiration)[draw, minimum width = 70,  fill = green!20, rounded corners] at (4.3,2.5) {\textcolor{green!50!black}{Inspiration}};
\path[draw,  line width = 25pt] (learn) -- (inspiration);};
\end{scope}

\end{tikzpicture}
\end{center}
}

\frame{\frametitle{Rolle des Lehrenden}
\framesubtitle{}
\begin{columns}[t] 
     \begin{column}[T]{6cm} 
     \vspace{1cm}
     	\begin{quote}
     		A teacher is never a giver of truth; he is a guide, a pointer to the truth that each student must find for himself.
     	\end{quote}
	\flushright Bruce Lee
	
     \end{column}
     	\begin{column}[T]{6cm} 
         	\begin{center}
            		\includegraphics[width=0.8\textwidth]{Bruce}
        		\end{center}
     \end{column}
 \end{columns}
}


\frame[allowframebreaks]{
\frametitle{Inhalt der Vorlesung}
\tableofcontents
}


\frame{\frametitle{Vorlesungsinhalte}
\framesubtitle{}
\begin{center}
\begin{tikzpicture}[scale = 0.95, small mindmap, concept color=gray!30]
\node [concept] {Schienen-fahrzeug-technik}
child [grow = 0]{node[concept] {Grundlagen}
child [grow = -30] {node[concept] {Systematik}}
child [grow = 30]{node[concept] {Normen und Regeln}}}
child [grow = 60]{node[concept] {Anfor-derungen}
child [grow = -20] {node[concept] {Requirements Engineering}}
child [grow = 20]{node[concept] {Lebenszyklus}}}
child [grow = 130]{node[concept] {Fahrzeuge}
child [grow = 30] {node[concept] {G\"uterwagen}}
child [grow = 150]{node[concept] {Personen-fahrzeuge}}
child [grow = 190]{node[concept] {Triebfahr-zeuge}}}
child [grow = 180]{node[concept] {Zugf\"order-technik}
child [grow = 160] {node[concept] {Spurf\"uhrung}}
child [grow = 200]{node[concept] {Rad-Schiene-Kontakt}}}
child [grow = 235]{node[concept] {Fahrzeug-konstruk-tion}
child [grow = 160] {node[concept] {Bauformen}}
child [grow = 200]{node[concept] {Mechanischer Aufbau}}}
child [grow = 300]{node[concept] {Laufwerk}
child [grow = 20] {node[concept] {Drehgestell}}
child [grow = 210]{node[concept] {Radsatz}}
child [grow = -20]{node[concept] {Federn und D\"ampfer}}
};
\end{tikzpicture}
\end{center}
}

\offslide{Fehlt etwas?}{Was k\"onnt Ihr noch gebrauchen? z.B. f\"ur die Railway Challenge, euer Mobilit\"atsfenster, ...}


\frame{\frametitle{Themenplan}
\framesubtitle{Das Lehrbuch von \emph{Janicki} et al. \citep{janicki} ist Pflichtlekt\"ure in diesem Modul, f\"ur weitere vertiefende Literatur siehe Literaturverzeichnis. Vorschl\"age f\"ur Themen der Seminararbeit sind willkommen!}
%\tiny
\hspace{1cm}
\begin{tabular}{|l|l|}
\hline
Thema & Kapitel aus \citep{janicki}\\ 
Pr\"aliminarien,  G\"uterwagen, Personenfahrzeuge& 6, 7.1\\ \hline
Einf\"uhrung Zugdynamik & 1.5.1, 1.5.2, 2.4 \\ \hline
Zugdynamik - Einf\"uhrung, Kraftschluss, Schlupf &  1.5.2 \\ 
Zugdynamik II - Fahrwiderstand, Zugkraft, Zugbremsung & 1.5, 5.2   \\ \hline
Einf\"uhrung Spurf\"uhrung & 1.4   \\ \hline
Kuppelsto{\ss}, Crash &   \\ \hline
Fahrzeugkonstruktion & 2.1.1 - 2.1.9  \\
Bauformen, Begrenzungen, Aufbau &  \\ 
Werkstoffe, F\"ugetechnik, &     \\
Brandschutz, Passive Safety &   \\ \hline
Laufwerke  & 2.2.1 - 2.2.13   \\
Einf\"uhrung, Radsatz, Drehgestell &  \\ 
Federung, D\"ampfung, Anbauten &  \\ \hline 
\end{tabular}
}

\frame{\frametitle{Semesterbegleitende Pr\"ufung, Praktikum}
\framesubtitle{}
\only<1>{
\begin{itemize}
\item Anhand Railway Challenge und \textit{Service Learning} f\"ur EVS
\begin{itemize}
	\item Aufgaben werden abgestimmt
\end{itemize}
\item Dokumentation durch technische Berichte
\item Gewichtung Berichte: in Summe 100\% der Modulnote
\end{itemize}}
\only<2>{
\begin{center}
\includegraphics[width = 7.5cm]{SFT1Plan}
\end{center}
}
}




%
%%System\"uberblick
%% !TEX root = SFV-14033_SFT1.tex
\section{System\"uberblick}
\frame{\sectionpage}

\frame{\frametitle{Definition Schienenfahrzeug}
\framesubtitle{Definition gem\"a{\ss} DIN 25003}
\begin{definition}[Schienenfahrzeuge \textit{rail vehicles}]
Spurgebundene Fahrzeuge, die mit Spurkranz versehenen R\"adern auf Gleisen, die aus Schienen einer bestimmten gleichbleibenden Spurweite gebildet sind, gef\"uhrt und getragen werden.
\end{definition}
Unterscheidung:
\begin{itemize}
	\item Eisenbahnfahrzeuge (gem\"a{\ss} AEG und EBO/ESBO)
	\item Strassenbahnen (gem\"a{\ss} PBefG und BOSTRAB)
	\item Nicht \"offentliche Bahnen (z.B. Werksbahnen) (gem\"a{\ss} BOA und EBOA)
\end{itemize}
}

% Din 25003
%Einführung in die Verkehrstechnik
%Zahlen und Fakten zum Verkehr
%Abgrenzung zur Fördertechnik
%Grundfunktionen des Schienenfahrzeugs
\subsection{Systematik}
\subsection{Systematik der Eisenbahnen}
\frame{\frametitle{Systematik des Eisenbahnverkehrs}
\framesubtitle{}
\begin{center}
\begin{tikzpicture}[thick]
		%\tiny
  \node[draw,rectangle, anchor = west] (a) {Eisenbahnen};
  \node[element, below of = a, left of = a, anchor = east] (b) {Personenverkehr};
  \node[element, below of = a, right of = a, anchor = west] (c) {G\"uterverkehr};
 
   \node[element, below of = b, left of = b, anchor = east] (e) {Fernverkehr}; 
  \node[element, below of= e](f) {Regionalverkehr};
  \node[element, below of=f] (g) {Intercity-Verkehr};
  \node[element, below of=g] (h) {Hochgeschwin-digkeitsverkehr};
  
  
  
 \node[element, below of = b, right of = b, anchor = west] (j) {Nahverkehr};
  \node[element, below of= j](k) {Stra{\ss}enbahn \\ \textit{Tram}};
  \node[element, below of=k] (l) {U-Bahn \\ \textit{Metro}};

%\node[element, below of=k] (l) {Triebzug \textit{Multiple Unit}};
 

    \draw[-] (a) -| (b);
     \draw[-] (a) -| (c);
    
    \draw[-] (b) -| (e);
    \draw[-] (b) -| (j); 
    
    \draw[-] (e) -- (f);
    \draw[-] (f) -- (g);
    \draw[-] (g) -- (h);
    
    \draw[-] (j) -- (k);
    \draw[-] (k) -- (l);
    
\end{tikzpicture}
\end{center}
}

\subsubsection{Systematik der Eisenbahnfahrzeuge}
\frame{\frametitle{Systematik der Eisenbahnfahrzeuge}
\framesubtitle{}
\begin{center}
\tikzstyle{element} = [draw,rectangle, align = center, text width = 2.5 cm, node distance = 1.25 cm]
\begin{tikzpicture}[thick]
		%\tiny
  \node[draw,rectangle] (a) {Eisenbahnfahrzeuge};
  \node[right of = a] (a1) {};
  \node[draw,rectangle, below of=a, minimum width = 4 cm, left of = a, anchor = east] (b) {Regelfahrzeuge};
  \node[left of = b] (b1) {};
  \node[right of = b] (b2) {};
  \node[draw,rectangle,below of=a, right of = a, anchor = west] (c) {Nebenfahrzeuge}; 
   \node[element, below of=b, left of = b, anchor = east] (d) {Wagen};
    \node[element, below of=d] (f) {G\"uterwagen \\ \textit{wagons}};
\node[element, below of=f] (g) {Reisezugwagen \textit{coaches}};
\node[element, below of=g] (h) {Sonstige \\ \textit{others}};

  \node[element, below of = b, right of = b, anchor = west] (e) {Triebfahrzeuge}; 
  \node[element, below of=e] (j) {Lokomotiven \textit{locomotives}};
\node[element, below of=j] (k) {Triebwagen \textit{railcar}};
\node[element, below of=k] (l) {Triebzug \textit{Multiple Unit}};
 

    \draw[-] (a) |- (b);
    \draw[-] (a) |- (c);
    \draw[-] (b) |- (d);
    \draw[-] (b) |- (e);
    
    \draw[-] (d) -- (f);
    \draw[-] (f)  -- (g);
    \draw[-] (g) -- (h);
    
    \draw[-] (e) -- (j);
    \draw[-] (j)  -- (k);
    \draw[-] (k) -- (l);
  
\end{tikzpicture}
\end{center}
}

%\offslide{Fahrzeuggattungen}{Tafelbild 6}




%
%% Grundlagen
%% !TEX root = SFV-14033_SFT1.tex
\section{Grundlagen}
\frame{\sectionpage}

\frame{\frametitle{Vergleich Straße-Eisenbahn \footnote{Nach Minde, IVE Hannover, 2007}}
\begin{small}
\begin{tabular}{|l|l|}
  \hline                       
Haftwert Gummi-Fahrbahn ($\mu_H \approx 0,9$) & Haftwert Stahl-Stahl ($\mu_H \approx 0,15$)  \\ 
Kleine Massen ($m_{abb} \approx 0,8 t$) & Große Radsatzmassen ($m_{abb} \approx 20 t$)  \\ 
Kurze Bremswege & Lange Bremswege  \\ \hline
Max. zwei gekuppelte Fahrzeuge & Zugbildung  \\ \hline
Ausweichen möglich & Spurführung  \\ \hline
Fahren auf Sicht & Signalisierung  \\ \hline
Relativer Bremswegabstand & Absoluter Bremswegabstand  \\ \hline
Rückkopplung Bremspedal & Verzögerte Bremswirkung  \\
  \hline  
\end{tabular}
\end{small}
}

\frame{\frametitle{Verhältnisse}
\begin{columns}[t] % contents are top vertically aligned
     \begin{column}[T]{6cm} % each column can also be its own environment
     \begin{itemize}
     \item \glqq Gute Verhältnisse\grqq \ der Eisenbahn vergleichbar mit \glqq Ausnahmesituationen\grqq \ auf der Straße
     \end{itemize}
     \end{column}
     \begin{column}[T]{5cm} % alternative top-align that's better for graphics
         \begin{center}
            \includegraphics[width=0.8\textwidth]{Bremsweg2}
        \end{center}
     \end{column}
     \end{columns}
}

\frame{\frametitle{Merkmale der Schienenfahrzeuge (teilweise nach \cite{schindler})}
\framesubtitle{}
\begin{center}
\scriptsize
\begin{tabular}{|p{.8cm}|p{1.4cm}|p{1.4cm}|p{1.4cm}|p{1.4cm}|p{1.5cm}|p{1.5cm}|}
\hline
 & Stra{\ss}enbahn & Stadtbahn & U-Bahn & S-Bahn & XMU CR & XMU HST \\ \hline
Gleis & Im Stra{\ss}enraum & Gro{\ss}teil eigener Gleisk\"orper & Eigener Gleisk\"orper & Vollbahngleis & Vollbahngleis & Vollbahngleis, z.T. HGV-Trassen \\ \hline
Kurven-radius & $\geq 15\mathrm{m}$ & $\geq 25\mathrm{m}$ & $\geq 90\mathrm{m}$ & $\geq 180\mathrm{m}$ & $\geq 625\mathrm{m}$ & $\geq 1800\mathrm{m}$ \\ \hline
Zugsi-cherung & Sicht & Sicht/Signale & Signale & Signale & Signale & F\"uhrerstands-signalisierung \\ \hline
$d_{Hp}$ & $(300 \ldots$ $ 600)\mathrm{m}$ & $(500 \ldots $ $ 800)\mathrm{m}$ & $(500 \ldots $ $ 1000)\mathrm{m}$ & $(750 \ldots $ $ 3000)\mathrm{m}$ & $(3 \ldots $ $20)\mathrm{km}$ & $\gg 20\mathrm{km} $ \\ \hline
L\"ange Fzg  & $(20 \ldots 53)\mathrm{m}$ & $(25 \ldots  40)\mathrm{m}$ & $(25 \ldots  40)\mathrm{m}$ & $(25 \ldots  40)\mathrm{m}$ & $\approx 26 \mathrm{m}$ & $ \approx 26 (28)\mathrm{m} $ \\ \hline
L\"ange Zug  & $\leq 75 \mathrm{m}$ & $ \leq 75\mathrm{m}$ & $\leq 120\mathrm{m}$ & $\leq 300 \mathrm{m}$ & $\leq 400 \mathrm{m}$ & $ \leq 400 \mathrm{m} $ \\ \hline
$v_{max}$  & $70 \mathrm{km/h}$ & $ 100\mathrm{km/h}$ & $100 \mathrm{km/h}$ & $140 \mathrm{km/h}$ & $\leq 189 \mathrm{km/h}$ & $ \geq 190 \mathrm{km/h} $ \\ \hline
$a_{max}$ & $\leq 1.5 \frac{\mathrm{m}}{\mathrm{s}^2}$ & $\leq 1.5 \frac{\mathrm{m}}{\mathrm{s}^2}$ & $\leq 1.3 \frac{\mathrm{m}}{\mathrm{s}^2}$ & $\leq 1.15 \frac{\mathrm{m}}{\mathrm{s}^2}$ & $\leq 1.15 \frac{\mathrm{m}}{\mathrm{s}^2}$ & $\leq 1.15 \frac{\mathrm{m}}{\mathrm{s}^2}$ \\ \hline
$b_{max}$ & $\leq 3 \frac{\mathrm{m}}{\mathrm{s}^2}$ & $\leq 3 \frac{\mathrm{m}}{\mathrm{s}^2}$ & $\leq 1.3 \frac{\mathrm{m}}{\mathrm{s}^2}$ & $\leq 1.0 \frac{\mathrm{m}}{\mathrm{s}^2}$ & $\leq 1.5 \frac{\mathrm{m}}{\mathrm{s}^2}$ & $\leq 1.5 \frac{\mathrm{m}}{\mathrm{s}^2}$ \\ \hline
$F_{L, test}$ & $\leq 300 \mathrm{kN}$ & $\leq 600 \mathrm{kN}$ & $\leq 800 \mathrm{kN}$ & $\leq 1500 \mathrm{kN}$ & $\leq 1500 \mathrm{kN}$ & $\leq 1500 \mathrm{kN}$ \\ \hline
\end{tabular}
\end{center}
}

\frame{\frametitle{Netz: Unterscheidung und Bedeutung}
\framesubtitle{}
\begin{itemize}
\item Hauptbahnen (Vollbahnen):
	\begin{itemize}
		\item F\"ur nationalen und internationalen Durchgangsverkehr
		\item Zweigleisig, i.d.R. elektrifiziert
		\item Ausbau f\"ur hohe Fahrgeschwindigkeiten: Kurvenradien, Fahrbahn, Zugsicherung
	\end{itemize}
	\item Stra{\ss}enbahnen:
	\begin{itemize}
		\item Definition s.o.
		\end{itemize}
	\item Nebenbahnen:
	\begin{itemize}
		\item Alle anderen
		\end{itemize}
\end{itemize}
}

\frame{\frametitle{Transeurop\"aisches Eisenbahnnetz}
\framesubtitle{Wichtig f\"ur Fahrzeugklassifizerung nach TSI}
\begin{center}
\includegraphics[width=0.75\textwidth]{TEN}
\end{center}
}

\subsection{Regulative und normative Grundlagen}
\frame{\frametitle{Regulative R\"aume am Beispiel des Systems Kupplung}
\framesubtitle{}
\begin{center}
\includegraphics[width=1.0\textwidth]{CouplerWorld}\\
{\color{blue!55!green} UIC/TSI} {\color{red!80!black} AAR} {\color{red!20!brown!60!black} GOST} {\color{violet} Mix}
\end{center}
}

\subsection{Aussenanschriften}
\frame{\frametitle{Europ\"aische Fahrzeugnummer (Reisezugwagen)}
\framesubtitle{}
\begin{center}
\begin{displaymath}
\underbrace{\mathsf{D-DB}}_{\text{\tiny Land, Fahrzeughalter}}
 \underbrace{50}_{\text{\tiny Fzg.-typ}}
 \underbrace{80}_{\text{\tiny Land d. Registrierung}}
 \underbrace{26}_{\text{\tiny Gattung}}
  - \underbrace{81}_{\tiny v_{max,}\text{\tiny Energieversorgung}}
  \underbrace{111}_{\text{\tiny Nummer innerhalb BR}}
  -\underbrace{9}_{\text{\tiny Pr\"ufziffer}}
\end{displaymath}
erg\"anzt um Bauart-/Gattungsbezeichnung.
\end{center}
}

\frame{\frametitle{Europ\"aische Fahrzeugnummer (Triebfahrzeuge)}
\framesubtitle{}
\begin{center}
\begin{displaymath}
\underbrace{9}_{\text{\tiny Selbstfahrend}}
\underbrace{1}_{\text{\tiny Tfz-Typ}}
 \underbrace{80}_{\text{\tiny Land d. Registrierung}}
 \underbrace{6185}_{\text{\tiny Baureihe}}
 \underbrace{750}_{\text{\tiny Nummer innerhalb BR}}
  -\underbrace{9}_{\text{\tiny Pr\"ufziffer}}
  \underbrace{\mathsf{D-BASF}}_{\text{\tiny Land und Tfz-Halter}}
\end{displaymath}
\end{center}
Triebfahrzeugtyp:
\begin{enumerate}
	\item Verschiedene
	\item Elektrische Lokomotive
	\item Diesellokomotive
	\item Elektrischer Triebzug (HGV)
	\item Elektrischer Triebzug (au{\ss}er HGV)
	\item Dieseltriebzug
	\item Spezieller Beiwagen
	\item Elektrische Rangierlokomotive
	\item Diesel-Rangierlokomotvie
	\item Sonderfahrzeug
\end{enumerate}
}


\frame{\frametitle{(Internationale) Verwendbarkeit}
\framesubtitle{}
\begin{itemize}
\item Internationale Verwendbarkeit laut RIC Raster, ggf. einzelne L\"ander
\item Angaben: $v_{max}$, L\"ander gem\"a{\ss} Vereinbarung, F\"ahrentauglichkeit, Stromarten zentrale Energieversorgung, weitere Ausr\"ustung
\end{itemize}
\begin{center}
            	\includegraphics[width=0.7\textwidth]{RICRaster}\\ \vspace{.2cm}	
		\includegraphics[width=0.7\textwidth]{RIC} \rotatebox{90}{{\tiny \color{gray} Quelle: Tobias K\"ohler}} 
		
        		\end{center}
}





%% G\"uterwagen
%% !TEX root = SFV-14033_SFT1.tex
\section{G\"uterwagen}
\frame{\sectionpage}


\frame{\frametitle{Zahlen zum G\"uterverkehr}
\framesubtitle{}
\begin{center}
\begin{tikzpicture}[scale = 1]
		\begin{axis}[width = 10 cm, ybar =2pt, bar width = 3pt,
    		xlabel={Jahr},
%    		ylabel={\color{blue!80!black}Mrd. tkm},
		%xtick=data,
		xtick={1997,2000,2005,2010,2015},
		x tick label style={rotate=45,anchor=east},
		ymin = 0, ymax = 125,
		legend style={at={(0.5,-0.30)},
        		anchor=north,legend columns=-1},
		/pgf/number format/.cd,
        		use comma,
        		1000 sep={},]
\addplot table {SGVMenge.dat};
\addplot table {SGVWagen.dat};
%\addplot table {data_d3.dat};
%\addplot table {data_d4.dat};
\legend{Mrd. tkm, 1000 G\"uterwagen}
\end{axis}
\end{tikzpicture}
\end{center}
}



\frame{\frametitle{Einf\"uhrung}
\framesubtitle{}
\begin{columns}[t] 
     \begin{column}[T]{5cm} 
     	\begin{itemize}
     		\item Gr\"o{\ss}te Gruppe an Fahrzeugen
		\item Universalwagen
		\begin{itemize}
		\item Standardisierte Verkehre
		\item z.B. Flachwagen
		\end{itemize}
		\item Sonderbauart 
		\begin{itemize}
		\item Bestimmte Verkehre
		\item z.B. Containertragwagen, Pkw-Transportwagen
		\end{itemize}     	
		\item H\"aufig im Privatbesitz
		\item Anspruchsvoll trotz einfacher Technik
		\end{itemize}
     \end{column}
     	\begin{column}[T]{7cm} 
         	%\begin{center}
		Regelbauart:
	     	\begin{tabular}{|l|c|}
		\hline 
		E & offene Wagen \\ \hline
		G & gedeckte Wagen \\ \hline
		K & Flachwagen (2 RS) \\ \hline
		O & gemischte Offen-Flachwagen \\ \hline
		R & Drehgestell-Flachwagen \\ \hline
		\end{tabular} \vspace{1mm}
		Sonderbauart:
		\begin{tabular}{|l|c|}
		\hline 
		F & offene Wagen \\ \hline
		H & gedeckte Wagen \\ \hline
		I & K\"uhlwagen \\ \hline
		L & Flachwagen mit unabh\"angigen RS \\ \hline
		S & Drehgestell-Flachwagen \\ \hline
		T & Wagen mit \"ofnungsf\"ahigem Dach \\ \hline
		U & Sonderwagen \\ \hline
		Z & Kesselwagen \\ \hline
		\end{tabular}
            	%\end{center}
     \end{column}
 \end{columns}
}

\offslide{Sammeln von Anforderungen}

\frame[allowframebreaks]{\frametitle{Anforderungen gem\"a{\ss} WAG TSI}
\framesubtitle{}
\begin{itemize}
\item Festigkeit gem\"a{\ss} EN12663-2
\begin{itemize}
	\item Zwei Kategorien: F-I: Allgemein, F-II: nicht ablaufen/absto{\ss}en
	\item L\"angsdruckkraft: F-I: 2000 kN, F-II: 1200 kN
	\item Zugkraft: 1000/1500 kN (je nach Anschlag)
\end{itemize}
\item Integrit\"at: bewegliche TEile sind gegen Positions\"anderungen gesichert
\item Begrenzungslinie abh\"angig vom Zielprofil
\item Radsatzlast gem\"a{\ss} EN 15228
\item Kompatibilit\"at mit Gleisfreimeldeanlagen
\begin{itemize}
	\item Gleisstromkreise
	\item Achsz\"ahler
	\item Kabelschleifen
\end{itemize}
\item Zustands\"uberwachung der Radsatzlager
\begin{itemize}
	\item Fahrzeugseitig 
	\item Streckenseitig (gem\"a{\ss} EN15437-1:2009)
\end{itemize}
\item Laufsicherheit
\begin{itemize}
	\item Sicherheit gegen Entgleisen unter Gleisverwindung
	\item Dynamisches Verhalten gem. EN14363 oder mittels validiertem Modell
\end{itemize}
\item Laufwerk
\begin{itemize}
	\item Festigkeit gem\"a{\ss} EN13749
	\item Forderungen an Rads\"aze und R\"ader gem\"a{\ss} WAG TSI
\end{itemize}
\item Bremse
\begin{itemize}
	\item Sicherheitsbetrachtung gem\"a{\ss} Common Safety Methods (CSM, (EG) Nr. 352/2009)
	\begin{itemize}
		\item Ausfall einer Einheit bei Mehrfachfehler
		\item Ausfall mehrerer Einheiten bei Einfachfehler
	\end{itemize}
	\item Bremsleistung
	\begin{itemize}
		\item Durch Berechnung gem\"a{\ss} EN14531-6
		\item Durch Versuch gem\"a{\ss} UIC 544-1
	\end{itemize}
	\item Feststellbremse
	\begin{itemize}
		\item Zustandsanzeige
	\end{itemize}
	\item W\"armekapazit\"at
	\begin{itemize}
		\item Dauerbremsung mit 45 kW (70 km/h, 40 km, i = 2,1\%)
	\end{itemize}
	\item Gleitschutz f\"ur Scheibenbremsen oder Klotzbremse mit $\mu_{m} > 0{,}12$
\end{itemize}
\item Umgebungsbedingungen
\begin{itemize}
	\item T1: -25 \si{\degreeCelsius} bis +40 \si{\degreeCelsius}
	\item T2: -40 \si{\degreeCelsius} bis +35 \si{\degreeCelsius}
	\item T3: -25 \si{\degreeCelsius} bis +45 \si{\degreeCelsius}
	\item Schnee, Eis und Hagel gem\"a{\ss} EN50125-1
\end{itemize}
\item Brandschutz
\begin{itemize}
	\item Abschirmung potenzieller Brandquelle von der Ladung
	\item Anforderungen an Materialien, Kabel und Fl\"ussigkeiten
\end{itemize}
\item Dokumentation
\begin{itemize}
	\item Betriebsunterlagen
	\item Instandhaltungsvorschriften
\end{itemize}
\end{itemize}
}

\frame{\frametitle{EG-Konformit\"at nach TSI WAG}
\framesubtitle{}
\begin{itemize}
\item F\"ur einige Elemente (Interoperabilit\"atskomponenten) wird von einer EG-Konformit\"at ausgegangen:
\begin{itemize}
	\item Einachsige Laufwerke: Doppelschakenaufh\"angung, Niesky 2, S 2000
	\item Drehgestelle mit zwei Rads\"atzen: Y25-Familie, zweiachsiges Lenkdrehgestell
	\item Dreiachsige Drehgestelle mit Schakenaufh\"angung
\end{itemize}
\item Auch f\"ur gewisse Materialien in Bezug auf Entflammbarkeit sowie Brandschutzw\"ande
\end{itemize}
}


%
%% Personenfahrzeuge
%% !TEX root = SFV-14033_SFT1.tex
\section{Personenfahrzeuge}
\frame{\sectionpage}

\frame{\frametitle{Einf\"uhrung}
\framesubtitle{Die Eisenbahn verkauft \emph{quality time}!}
\begin{itemize}
\item Anspruchsvolle Fahrg\"aste
\begin{itemize}
	\item Verschiedene Anspr\"uch je nach Verkehrsart
	\item Art und Ausstattung an Gattungsbezeichnung zu erkennen
\end{itemize}
\item Umsetzung als Wagen oder Triebzug
\item Wichtige Aspekte:
\begin{itemize}
	\item Inneneinrichtung und Grundriss
	\item Zugang
	\item Ausstattung
	\item Energieversorgung
	\item Fahrkomfort
	\item Fahrgaststr\"ome
	\item Reisegeschwindigkeit
\end{itemize}
\item In verschiedenen Kulturen verschieden Akzeptanz des Bahnverkehrs!
\end{itemize}
}

\frame{\frametitle{Gattungssystematik}
\framesubtitle{}
\begin{center}
%\tiny
\begin{tabular}{|c|c|c|c|}
\hline
\multicolumn{2}{|c|}{Gattungsbuchstabe} & \multicolumn{2}{|c|}{Kennbuchstabe} \\ \hline
A & Sitzwagen 1. Klasse & m & Reisezugwagen oder \\
& & & Wagen eines Triebzugs \\ \hline
AB & Sitzwagen 1. und 2. Klasse & n/y & Nahverkehrswagen \\ \hline
B & Sitzwagen 2. Klasse & x & S-Bahn-Wendezugwagen \\ \hline
D & Gep\"ackwagen & f & mit F\"uhrerraum \\ \hline
D... & Doppelstockwagen & p & klimatisiert, Gro{\ss}raum \\ \hline
...R & mit K\"uche, Bistro & o & vergr\"o{\ss}erte Abteile \\ \hline
...D & mit Gep\"ackabteil & b & Rollstuhleinrichtungen \\ \hline
WL & Schlafwagen & d & mit Mehrzweckraum \\ \hline
WR & Speisewagen & r & mit Rapidbremse \\ \hline
 &  & h/z & Energieversorgung \\ \hline
\end{tabular}
\end{center}
}

%\subsection{Inneneinrichtung}

\frame{\frametitle{Schutzziel gem. TSI}
\framesubtitle{}
\begin{itemize}
\item „Die für die Betätigung durch die Fahrgäste vorgesehenen Einrichtungen müssen so konzipiert sein, dass sie deren Sicherheit nicht gefährden, wenn sie in einer voraussehbaren Weise betätigt werden, die den angebrachten Hinweisen nicht entspricht.“
\end{itemize}
}


\frame{\frametitle{Inneneinrichtung}
\framesubtitle{}
\begin{itemize}
\item Unterschiedliche Bed\"urfnisse in den verschiedenen Verkehrsarten
\item H\"aufig sehr detailliert Inhalt von Verkehrsausschreibungen
\begin{itemize}
	\item Transportm\"oglichkeiten (Fahrrad, Kinderwagen, Rollst\"uhle,...)
	\item Sitzpl\"atze, Tische
	\item \"Uberwachungssysteme (CCTV)
\end{itemize}
\item Einstieg
\begin{itemize}
	\item Fernverkehr: Wagenende
	\item Regional-, Nahverkehr: Dritteleinstieg (oder h\"aufiger)
\end{itemize}
\item Sitzanordnungen
\begin{itemize}
	\item Abteil: 4, 5, 6 Sitze je Abteil, Seitengang
	\item Gro{\ss}raum: i.d.R. 3 oder 4 Sitzpl\"atze je Reihe, Mittelgang
	\item In UK, China: 5 Sitzpl\"atze je Reihe
\end{itemize}
\item Sitzplatzanzahl: Effizienz dominiert
\end{itemize}
}

%\subsection{Barrierefreiheit}
\frame[allowframebreaks]{\frametitle{Barrierefreiheit}
\framesubtitle{Transversale PRM TSI \textit{people with reduced mobility} stellt Anforderungen dar.}
\begin{itemize}
\item Definition People with reduced mobility
\begin{itemize}
	\item Personen, die mit der Nutzung von Eisenbahnen (Fahrzeuge und Infrastruktur) Schwierigkeiten haben
\end{itemize}
\item Au{\ss}ent\"ur:
\begin{itemize}
	\item Kontrast zum Fahrzeug, Bedienung auf oder neben dem T\"urblatt, Sicht- und H\"orbare Warnung bei Bet\"atigung
	\item Lichte Weite mindestens 800 mm (HST), mindestens 1000 mm (CR)
\end{itemize}
\item Zustiegshilfe
\begin{itemize}
	\item W\"unschenswert: angepasste Fahrzeuge f\"ur Infratruktur
	\item Sonst: Rampen, \"Uberfahrbr\"ucken etc.
\end{itemize}
\item Inneneinrichtung
\begin{itemize}
	\item Verf\"ugbarkeit von Haltegriffen, Vorrangsitzen (10\%)
	\item Rollstuhlp\"atze: 1 ($L_{Zug} < 30$ m) bis 4 ($L_{Zug} > 300$ m)
	\item Hilferufvorrichtung
	\item Rampen eingeschr\"ankt zul\"assig
	\item Haltestangen, D = (30...40) mm
\end{itemize}
\item Toiletten
\begin{itemize}
	\item Vorhandensein einer Universaltoilette
\end{itemize}
\item Fahrgastinformation:
\begin{itemize}
	\item Piktogramme (max. 5 zusammen)
	\item Taktile Informationen
	\item Displays etc. von 51\% der Fahrgastpl\"atze und allen Rollstuhlpl\"atzen lesbar
\end{itemize}
\end{itemize}
}

%\subsection{Energieversorgung}
\frame{\frametitle{Energieversorgung}
\framesubtitle{}
\begin{itemize}
\item In Wagen:
\begin{itemize}
	\item Dominierend: Zugsammelschiene gem\"a{\ss} UIC 552
	\item Verschiedene Spannungen und Frequenzen, z.B.:
	\begin{itemize}
		\item AC 1000 V 16,7 Hz
		\item AC 1500 V 50 Hz
		\item DC 1500 V
		\item DC 3000 V
	\end{itemize}
	\item Strom: (800...1000) A (je Kupplung 600 A)
	\item Stromart erfordert Gl\"attung/Wechselrichtung
	\item Vereinzelt Achgeneratoren
\end{itemize}
\item In Triebz\"ugen:
\begin{itemize}
	\item Verbindung im Rahmen der Fahrzeugverdrahtung
	\item Bordnetzspannung 24 V, 72 V, 110 V (je -30\%/+25\%)
\end{itemize}
\end{itemize}
}

\frame{\frametitle{T\"uren und T\"ursteuerung}
%\framesubtitle{``Die beiden gef\"ahrlichen Schnittstellen mit der Bev\"olkerung sind Bahn\"uberg\"ange und T\"uren''}
\begin{columns}[t] 
     \begin{column}[T]{6cm} 
     	\begin{itemize}
     		\item Wichtige Aspekte:
		\begin{itemize}
		\item \"Offnungsweite
		\item Druckert\"uchtigung
		\item Festigkeit (insb. HST)
		\item Sicherheit
		\end{itemize}
		\item Bauarten:
		\begin{itemize}
		\item Drehfaltt\"ur
		\item Schwenkschiebet\"ur in verschiedenen Bauarten
		\end{itemize}
		\item T\"ursteuerung:
		\begin{itemize}
		\item Verschiedene Verfahren (Automatisierung):
		\begin{itemize}
		\item T\"ursicherung
		\item TB 0%: %T\"urblockierung ab 0 km/h
		\item SAT%: Selbstabfertigung durch Tf
		\item TAV%: Technikbasiertes Abfertigungsverfahren 
		\end{itemize}
		\end{itemize}
     	\end{itemize}
     \end{column}
     	\begin{column}[T]{6cm} 
         	\begin{center}
            		\includegraphics[width=0.4\textwidth]{Drehfalttuer} \rotatebox{90}{\color{gray} \tiny Quelle: Wikimedia/LosHawlos}\\
		\includegraphics[width=0.8\textwidth]{SST} \rotatebox{90}{\color{gray} \tiny Quelle: Wikimedia/Lief J\"orgensen}
        		\end{center}
     \end{column}
 \end{columns}
}

\frame{\frametitle{Klimatisierung}
\framesubtitle{Die Aufgaben Heizen, Bel\"uften und Klimatisieren werden h\"aufig integriert (HVAC).}
\begin{itemize}
\item Ausf\"uhrungen:
\begin{itemize}
	\item Heute dominierend: elektrische Energieversorgung
	\item Noch im Bestand: Dampf/elektrische Heizungen, \"Olheizungen
	\item F\"ur K\"uhlung: K\"uhlmittel- und Kaltluftanlagen 
\end{itemize}
\item Aufgaben:
\begin{itemize}
	\item Heizen: Innenraumtemperatur auf bestimmtem Niveau halten
	\item Bel\"uftung: ben\"otigte Luftmenge zuf\"uhren
	\item Klimatisieren: Innenraumtemperatur auf bestimmtem Niveau halten
\end{itemize}
\item Herausforderungen:
\begin{itemize}
	\item Gro{\ss}e Fahrzeugfl\"achen und -scheiben
	\item Hohe, schwankende Personenzahlen
	\item Installationsraum
	\item T\"ur\"offnung
	\item Feuchtigkeitszufuhr (nasse Reisende)
	\item Zugfreiheit
\end{itemize}
\end{itemize}
}

\frame{\frametitle{Fahrgastnotruf}
\framesubtitle{Der Fahrgastnotruf l\"ost die Notbremse bei TSI-konformen Fahrzeugen ab.}
\begin{columns}[t] 
     \begin{column}[T]{6cm} 
     	\begin{itemize}
     		\item Ausstattung:
		\begin{itemize}
		\item Jedes Abteil, Vorr\"aume und alle anderen abgetrennten Bereiche ausser Toiletten und \"Uberg\"ange
		\item Sichtbar und gekennzeichnet
		\end{itemize}
		\item Alarm kann nicht abgebrochen werden
		\item Alarm wird Tf visuell und akustisch angezeigt
		\item Tf kann best\"atigen, dies wird Fahrg\"asten mitgeteilt
		\item Kommunikation mit Tf
		\item R\"ucksetzung durch Zugpersonal
     	\end{itemize}
     \end{column}
     	\begin{column}[T]{6cm} 
         	\begin{center}
            		\includegraphics[width=0.65\textwidth]{Notbremse}
        		\end{center}
     \end{column}
 \end{columns}
}

\frame{\frametitle{Fahrgastinformationssysteme}
\framesubtitle{}
\begin{itemize}
\item Aufgaben:
\begin{itemize}
	\item Information des Reisenden: Zuglauf, n\"achster HAlt, etc.
	\item Kommunikation (betrieblich und \"offentlich, Mobilfunk-Repeater, WLAN, ...)
	\item Unterhaltung
	\item Kommunikation im Notfall (siehe SFT2: Notbremsanforderung)
\end{itemize}
\item Umsetzung:
\begin{itemize}
	\item Anzeigen
	\item Elektroakustische Anlage (ELA)
\end{itemize}
\end{itemize}
}

\frame{\frametitle{Fahrgastinformationssysteme}
\framesubtitle{}
\begin{columns}[t] 
     \begin{column}[T]{6cm} 
     	\begin{itemize}
     		\item Aufgaben:
\begin{itemize}
	\item Information des Reisenden: Zuglauf, n\"achster HAlt, etc.
	\item Kommunikation (betrieblich und \"offentlich, Mobilfunk-Repeater, WLAN, ...)
	\item Unterhaltung
	\item Kommunikation im Notfall (siehe SFT2: Notbremsanforderung)
\end{itemize}
\item Umsetzung:
\begin{itemize}
	\item Anzeigen
	\item Elektroakustische Anlage (ELA)
\end{itemize}

     	\end{itemize}
     \end{column}
     	\begin{column}[T]{6cm} 
         	\begin{center}
            		\includegraphics[width=0.8\textwidth]{FIS}
        		\end{center}
     \end{column}
 \end{columns}
}



% Anforderungen
% !TEX root = SFT1-Skript.tex
% !TEX root = SFT1-Folien.tex
\section{Requirements Engineering}
\lehrtext{
In diesem Abschnitt der Veranstaltung sollt ihr lernen,
\begin{itemize}
	\item wo die Anforderungen an Schienenfahrzeuge herkommen,
	\item wie man systematisch mit ihnen arbeitet und
	\item welche Pr\"ufschritte euch (auch in Industrieprojekten) erwarten.
\end{itemize}
}

\frame{\frametitle{Warum Requirements Engineering (RE)?}
\lehrtext{Requirements Engineering befasst sich mit dem systematischen Erfassen, Umsetzen und Pr\"ufen von Anforderungen im Entwicklungsprozess. Durch striktes RE kann man Projektrisiken erheblich reduzieren und evtl. auch Kosten sparen.}
\begin{itemize}
\item Qualit\"at: Qualit\"at ist das Ma{\ss} der Erf\"ullung der Anforderungen an ein Produkt.
\item Kosten- und Termintreue
\item Einbindung der Stakeholder (Anspruchsteller)
\item Systematisierung der Beschaffung und des Engineerings
 
\end{itemize}
}

\frame{\frametitle{Key-Aspects of Requirements Engineering}
\lehrtext{
Was RE vom sonstigen Entwicklungsprozess unterscheidet ist vornehmlich, dass:
\begin{itemize}
	\item alle, die im Projekt Anspr\"uche haben (``Stakeholder'') k\"onnen sich einbringen,
	\item Pr\"ufungen werden als Reviews mit den Stakeholdern durchgef\"uhrt und
	\item jede Anforderung ist von Anforderung bis zum Nachweis der Erf\"ullung nachvollziehbar.
\end{itemize}
Beim dritten Punkt wird sichergestellt, dass eine geschlossene Kette von Verweisen von der Anforderung \"uber alle nachgelagerten Dokumente (Pflichtenheft, Konstruktionsunterlagen, Test-Spec...) f\"uhrt, die sicherstellt, dass diese Anforderung umgesetzt und erf\"ullt wurde. \href{https://de.wikipedia.org/w/index.php?title=Rückverfolgbarkeit_(Anforderungsmanagement)&oldid=197422472}{Traceability bei Wikipedia}
}
\begin{itemize}
\item Stakeholder Involvement
\item Technical Reviews
\item Traceability
\end{itemize}
}

%\offslide{Generisches Phasenmodell}{Modell einer beliebigen Phase eines Entwicklungsprozesses}

\frame{\frametitle{Generisches Phasenmodell}
\lehrtext{
Jede Phase eines Entwicklungsprozesses kann (sollte!) wie folgt abgebildet sein, damit die Qualit\"at des Entwicklungsschritts sichergestellt werden kann.

}
\begin{columns}[t] 
     \begin{column}[T]{5cm} 
     \textbf{F\"ur jede Phase festzulegen:}
     	\begin{itemize}
     		\item Purpose
		\item Inputs
		\item Entry Criteria
		\item Roles
		\item Verification steps
		\item Outputs
		\item Exit criteria
		\item Resources
		\item Management review activities
     	\end{itemize}
     \end{column}
     	\begin{column}[T]{7cm} 
         	\begin{center}
            		\includegraphics[width=1.0\textwidth]{Phase.png}
        		\end{center}
     \end{column}
 \end{columns}
}

%\offslide{V-Modell f\"ur Requirements Engineering}

\frame{\frametitle{V-Modell f\"ur Requirements Engineering}
\lehrtext{
Die einzelnen Phasen k\"onnen dem V-Modell entnommen werden (es empfiehlt sich sogar!). Bitte beachtet im Bild die Verbindungen zwischen absteigendem Pfad (Entwicklungsdokumente) und aufsteigendem Pfad (Pr\"ufdokumente): hier hat die Organisation die M\"oglichkeit, sich weiterzuentwickeln. \href{https://de.wikipedia.org/wiki/V-Modell_(Entwicklungsstandard)}{V-Modell bei Wikipedia}
}
         	\begin{center}
            		\includegraphics[width=0.9\textwidth]{VModel}\source{US DoT}
        		\end{center}
}

\frame{\frametitle{Requirements Analysis}
\lehrtext{Im ersten Schritt geht es um das Ermitteln der System Level Requirements, also der sehr hoch angesiedelten Anforderungen. Das h\"ort sich einfacher an, als es ist: viele Kunden wissen gar nicht, was sie brauchen. Daher liest man h\"aufig auch den Term ``Elicit Requirements (Anforderungen hervorlocken)'' daf\"ur.

Der erste Meilenstein fragt daher ab, ob beide Seiten die Anforderungen verstanden haben. Hier geht es auch um rudiment\"are Dinge wie den Ausgabestand des Lastenhefts.

}
\begin{columns}[t] 
     \begin{column}[T]{6cm} 
     \textbf{Leitfragen:}
     	\begin{itemize}
		\item What are the stakeholders?
     		\item What is the system to do?
		\item How well it is to do it?
		\item Under what conditions?
     	\end{itemize}
	\textbf{Typischer Meilenstein: Initial Design Review (IDR)}
     \end{column}
     	\begin{column}[T]{6cm} 
         	\begin{center}
            		\includegraphics[width=0.95\textwidth]{Phase}
        		\end{center}
     \end{column}
 \end{columns}
}

%\offslide{Erg\"anzen des generischen Phasenmodells}{}

\frame{\frametitle{System Specification}
\lehrtext{Im n\"acshten Level geht es um den Entwurf auf einer hohen Abstraktionsebene, also um die Architektur, L\"osungen und den Zuschnitt der Subsysteme.
}
\begin{columns}[t] 
     \begin{column}[T]{6cm} 
     \textbf{Leitfragen:}
     	\begin{itemize}
		\item Is the required system feasible?
     		\item What are system and subsystem borders?
		\item What are associated costs/lead times/risks?
		\item How can the risk be reduced?
		\item Which system integration steps are necessary?
     	\end{itemize}
	\textbf{Typischer Meilenstein: Preliminary Design Review (PDR)}
     \end{column}
     	\begin{column}[T]{6cm} 
         	\begin{center}
            		\includegraphics[width=0.95\textwidth]{Phase}
        		\end{center}
     \end{column}
 \end{columns}
}

%\offslide{Erg\"anzen des generischen Phasenmodells}%{Durchf\"uhrung in der \"Ubung}

\frame{\frametitle{Subsystem Design}
\lehrtext{
In diesem Schritt wird die Feinstruktur des Systems entwickelt (``Lower Level Design''), also die Architektur der Subsysteme, Detaill\"osungen und der Zuschnitt der Module.}
\begin{columns}[t] 
     \begin{column}[T]{6cm} 
     \textbf{Leitfragen:}
     	\begin{itemize}
		\item What are the subsystem requirements?
		\item Make or Buy?
		\item Which deliverables (e.g. documentation) are requested?
		\item What is the suitable subsystem structure?
     	\end{itemize}
	\textbf{Typischer Meilenstein: Critical Design Review (CDR)}
     \end{column}
     	\begin{column}[T]{6cm} 
         	\begin{center}
            		\includegraphics[width=0.95\textwidth]{Phase}
        		\end{center}
     \end{column}
 \end{columns}
}

%\offslide{Erg\"anzen des generischen Phasenmodells}%{Durchf\"uhrung in der \"Ubung}

\frame{\frametitle{Module Design}
\lehrtext{Jetzt geht es ins Eingemachte: im Module Design entwickelt man die ``Build to Specifications'', also Zeichnungen, Schemata usw., nach denen dann wirklich gefertigt bzw. beschafft wird. Dementsprechend muss das dann auch ordentlich \"uberpr\"uft werden - in der Regel ganz klassisch im Vier-Augen-Prinzip, also Pr\"ufung und Freigabe

}
\begin{columns}[t] 
     \begin{column}[T]{6cm} 
     \textbf{Leitfragen:}
     	\begin{itemize}
		\item How can the module be realised efficiently?
		\item What are critical characteristics of the module and its parts?
		\item Can service proven modules be used or adapted?
     	\end{itemize}
     \end{column}
     	\begin{column}[T]{6cm} 
         	\begin{center}
            		\includegraphics[width=0.95\textwidth]{Phase}
        		\end{center}
     \end{column}
 \end{columns}
}

%\offslide{Erg\"anzen des generischen Phasenmodells}

%ISO15288


%Fahrzeugkonstruktion
% !TEX root = SFV-14033_SFT1.tex
\section{Fahrzeugkonstruktion}
\sectionpage

\subsection{Bauformen}
\subsectionpage
\frame{\frametitle{Konstruktionsprinzipien der Wagenk\"asten}
\framesubtitle{}
\begin{columns}[t] 
\begin{column}[T]{.5cm}
\end{column} 
     \begin{column}[T]{5.5cm} 
     \textbf{Differenzialbauweise}
     	\begin{itemize}
     		\item Fertigung aus Halbzeugen:
		\begin{itemize}
		\item Einzelteile einfach geformt
		\item Formgebung durch F\"ugen und Umformen
		\end{itemize}
     	\end{itemize}
	\textbf{Integralbauweise}
     	\begin{itemize}
     		\item Fertigung aus komplex geformten Elementen:
		\begin{itemize}
		\item z.B. Strangpressprofile
		\item Formgebung durch F\"ugen und Zerspanen
		\end{itemize}
     	\end{itemize}
	\textbf{Tragfunktion}
     	\begin{itemize}
     		\item Tragendes Untergestell
		\item Selbsttragender Wagenkasten
	\end{itemize}
     \end{column}
     	\begin{column}[T]{6cm} 
         	\begin{center}
            		\includegraphics[width=0.8\textwidth]{Integral}\rotatebox{90}{\tiny \color{gray} Quelle: Siemens Pressebild}\\
		Wagenkasten in Integralbauweise
        		\end{center}
     \end{column}
 \end{columns}
}


\subsection{Begrenzungen}
\subsectionpage

\frame{\frametitle{Lichtraumprofil streckenseitig}
\framesubtitle{}
\begin{columns}[t] 
     \begin{column}[T]{6cm} 
     	\begin{itemize}
		\item Streckenseitiges Lichtraumprofil muss ber\"ucksichtigen
		\begin{itemize}
		\item Beladungszust\"ande
		\item Dynamische Bewegungen:
		\begin{itemize}
		\item Ein-/Ausfedern
		\item Wanken
		\item Nicken
		\end{itemize}
		\item Bogenfahrt
		\item Kompatibilit\"at mit anderen Fahrzeugen
		\end{itemize}
     		\item Deutsches Regelprofil: G2
		\item Europ\"aisch: G1
		\item Betrieblich Ladema{\ss}\"uberschreitungen m\"oglich
     	\end{itemize}
     \end{column}
     	\begin{column}[T]{6cm} 
         	\begin{center}
			\only<1>{
            		\includegraphics[width=0.9\textwidth]{G2Strecke}\source{Quelle: Christian Lindecke} \\
			\small Lichtraumprofil G2 gem\"a{\ss} EBO}
			\only<2>{\includegraphics[width=0.9\textwidth]{TGCan}\\
			\small Kanadisches Lichtraumprofil}
        		\end{center}
     \end{column}
 \end{columns}
}

\frame{\frametitle{Fahrzeugbegrenzung: Querschnitt}
\framesubtitle{}
         	\begin{center}
            		\includegraphics[width=0.8\textwidth]{G1G2}\source{Quelle: Christian Lindecke}
        		\end{center}
     }


\offslide{Breiteneinschr\"ankung und Lichtraumbedarf}

\frame{\frametitle{Radsatzlasten und Meterlasten}
\framesubtitle{}
\begin{columns}[t] 
     \begin{column}[T]{6cm} 
     	\begin{itemize}
     		\item Beschr\"ankung der Radsatzlast:
		\begin{itemize}
		\item Gem\"a{\ss} Streckenkategorie
		\item Normativ, z.B. TSI Loc\&Pas (f\"ur HGV), EN 15528
		\end{itemize}
		\item Beschr\"ankung der Streckenlast
		\begin{itemize}
		\item z.B. f\"ur Br\"uckenbauwerke, Oberbau
		\end{itemize}
     	\end{itemize}
     \end{column}
     	\begin{column}[T]{6cm} 
         	\begin{center}
		\begin{tabular}{|c|c|c|}
		\hline
			Klasse & Radsatzlast & Meterlast \\ \hline
			A & 16 t & 5{,}0 t/m \\ \hline
			B1 & 18 t & 5{,}0 t/m \\ \hline
			B2 & 18 t & 6{,}4  t/m \\ \hline
			C2 & 20 t & 6{,}4  t/m \\ \hline
			C3 & 20 t & 7{,}2  t/m \\ \hline
			C4 & 20 t & 8{,}0  t/m \\ \hline
			D2 & 22{,}5 t & 6{,}4  t/m \\ \hline
			D3 & 22{,}5 t & 7{,}2  t/m \\ \hline
			D4 & 22{,}5 t & 8{,}0  t/m \\ \hline
			E4 & 25 t & 8{,}0  t/m \\ \hline
			E5 & 25 t & 8{,}8  t/m \\ \hline
		\end{tabular}
        		\end{center}
     \end{column}
 \end{columns}
}

\offslide{L\"angen- und Gewichtseinschr\"ankungen}

\subsection{Wagenkastenrohbau}
\subsectionpage
\frame{\frametitle{Anforderungen an den Wagenkasten \textit{car body}}
\framesubtitle{}
\begin{itemize}
\item Festigkeit (EN 12663):
\begin{itemize}
	\item Zug-/Druckkr\"afte im Zugverband
	\item Crash-Szenarien (EN 15227)
	\item Drucks\"o{\ss}e, Druckdichtigkeit
	\item Durchbiegung unter Beladung
	\item Schwingungen
\end{itemize}
\item Kunden-/ betriebliche Anforderungen
\begin{itemize}
	\item Lebensdauer
	\item Reparaturfreundlichkeit, Ersatzteilverf\"ugbarkeit
	\item Geringe Masse
	\item Design
	\item Entsorgung/Recycling
\end{itemize}
\item Normative/gesetzliche Anforderungen
\begin{itemize}
	\item Brandschutz (DIN 5510, EN 45545, ...)
	\item Material (EG 1907/2006 REACH)
	\item Crash und Festigkeit s.o.
\end{itemize}
\item Systemimmanente Anforderungen (Schwingungen, elastische Verformung,...)
\end{itemize}
}



\frame{\frametitle{Leichtbau der Wagenk\"asten}
\framesubtitle{}
\begin{itemize}
\item Alle Elemente an Aufnahme der Beanspruchungen beteiligen
\item Gut (leicht) ertragbar:
\begin{itemize}
	\item Zug- und Druckkr\"afte
\end{itemize}
\item Mit zus\"atzlichem Material ertragbar:
\begin{itemize}
	\item Torsions- und Biegemomente
\end{itemize}
\item H\"oherfeste Materialien werden z\"ogerlich angenommen
\begin{itemize}
	\item Bedenken bei Wartbarkeit und Lebensdauer
\end{itemize}
\end{itemize}
\begin{center}
\includegraphics[width=0.8\textwidth]{Class66}
\end{center}
}

\frame{\frametitle{Werkstoffe f\"ur Wagenk\"asten}
\framesubtitle{}
\begin{itemize}
\item Stahl:
\begin{itemize}
	\item Klassisch eingesetzt: Baust\"ahle S235, S355
	\item Ebenfalls anzutreffen: Edelst\"ahle, z.B. X5CrNi18-10
	\item Gut zu f\"ugen und umzuformen
	\item Dauerfestigkeit und elastisch/plastisches Verhalten gutm\"utig
\end{itemize}
\item Aluminium:
\begin{itemize}
	\item Geringere Dichte, geringerer E-Modul
	\item Dauerfestigkeitsgrenze wenig ausgepr\"agt
	\item Schwei{\ss}n\"ahte wenig erm\"udungsfest
	\item F\"ugeverfahren erfordern getrennte Behandlung von Stahl
\end{itemize}
\item Kunststoffe:
\begin{itemize}
	\item In der Regel faserverst\"arkt (GFK, CFK)	
	\item Erm\"oglichen Integralbauweise und Funktionsintegration
	\item Auch als Sandwichmaterialien
\end{itemize}
\item Waben und Schaummaterialien:
\begin{itemize}
	\item Eingesetzt im Deformationsbereich
\end{itemize}
\end{itemize}
}

\frame{\frametitle{Hauptbaugruppen des Rohbaus}
\framesubtitle{}
\begin{columns}[t] 
     \begin{column}[T]{6cm} 
     	\begin{itemize}
     		\item Untergestell
		\begin{itemize}
		\item (Mittel/Aussen-) Langtr\"ager
		\item Quertr\"ager
		\end{itemize}
		\item Seitenw\"ande
		\begin{itemize}
		\item Druckwechselbelastung
		\end{itemize}
		\item Dach
		\begin{itemize}
		\item Wasserablauf
		\end{itemize}
		\item Endw\"ande
		\begin{itemize}
		\item Schnittstellen, Crash
		\end{itemize}
		\item Kopfmodule
		\begin{itemize}
		\item Vorfertigung, Schnittstellen, Crash
		\end{itemize}
     	\end{itemize}
     \end{column}
     	\begin{column}[T]{6cm} 
         	\begin{center}
            		\includegraphics[width=0.8\textwidth]{FrontNose}
		\rotatebox{90}{\tiny \color{gray}{Quelle: Voith Pressebild}}
        		\end{center}
     \end{column}
 \end{columns}
}

\frame{\frametitle{Prozess Wagenkastenfertigung}
\framesubtitle{Bei allen Schritten zu beachten: Teils extremer Verzug durch W\"armeeinleitung}
\begin{columns}[t] 
     \begin{column}[T]{6cm} 
     	\begin{enumerate}
     		\item Einzelteilfertigung
		\begin{itemize}
		\item Schneiden, Schwei{\ss}nahtvorbereitung, Kanten, etc.
		\end{itemize}
		\item Baugruppenfertigung
		\begin{itemize}
		\item Schwei{\ss}en, evtl. Bearbeitung
		\item Hand- oder Roboterschwei{\ss}en je nach Naht
		\item Vermessung
		\end{itemize}
		\item Wagenkastenaufbau
		\begin{itemize}
		\item Vorsprengung bei statischer Durchbiegung
		\item Dichtigkeitspr\"ufung
		\end{itemize}
		\item Richten
		\item Sandstrahlen
     	\end{enumerate}
     \end{column}
     	\begin{column}[T]{6cm} 
         	\begin{center}
            		\includegraphics[width=0.8\textwidth]{Rohbau}\rotatebox{90}{\tiny \color{gray} Quelle: Siemens Pressebild}\\  
		Pr\"ufen der Aussenkontur     		
		\end{center}
     \end{column}
 \end{columns}
}


%Laufwerk
% !TEX root = SFV-14033_SFT1.tex
\section{Laufwerk (Fahrwerk)}
\frame{\sectionpage}

\frame{\frametitle{Grunds\"atzliche Anforderungen}
\framesubtitle{}
\begin{itemize}
\item \"Ubertragung und Ausgleich der Vertikallasten zwischen Rad und Schiene
\item Spurf\"uhrung des Fahrzeugs
\item \"Ubertragung und Begrenzung der dynamischen Kr\"afte, aufgrund von:
\begin{itemize}
		\item Gleislagefehlern
		\item B\"ogen
		\item Weichen
		\item Dynamik zwischen den Fahrzeugen
		\end{itemize} 
\item Wirksame D\"ampfung von angeregten Schwingungen
\item \"Ubertragung von Traktions- und Bremskr\"aften
\begin{center}
\includegraphics[width=0.6\textwidth]{Bogiediagram} \source{Quelle: Christophe Jacquet}
 \end{center}
\end{itemize}
}

\frame{\frametitle{Anatomie der Eisenbahndrehgestelle \textit{bogies}}
\framesubtitle{}
\begin{columns}[t] 
     \begin{column}[T]{5cm} 
     	\begin{itemize}
     		\item Rads\"atze \textit{wheelset}
		\item R\"ader \textit{wheels}
		\item Radsatzlager \textit{axlebox}
		\item Radsatzaufh\"angung \textit{suspension}
		\begin{itemize}
		\item Federn
		\item D\"ampfer
		\end{itemize}
		\item Begrenzungen und Anschl\"age
		\item Wagenkastenanbindung
		\item Drehgestellrahmen \textit{bogie frame}
     	\end{itemize}
     \end{column}
     	\begin{column}[T]{7cm} 
         	\begin{center}
            		\includegraphics[width=0.9\textwidth]{SchemaDG}\source{Quelle: Partim}
        		\end{center}
     \end{column}
 \end{columns}
}

\frame{\frametitle{Rads\"atze}
\framesubtitle{}
\begin{columns}[t] 
     \begin{column}[T]{6cm} 
     	\begin{itemize}
     		\item Unterscheidung:
		\begin{itemize}
		\item Innen-/Aussenlagerung
		\item Bremse
		\begin{itemize}
		\item Klotzbremse
		\item Radbremsscheibe
		\item Wellenbremsscheibe
		\end{itemize}
		\item Antriebe
		\begin{itemize}
		\item Symmetrisch
		\item Asymmetrisch
		\end{itemize}
		\end{itemize}
     	\end{itemize}
     \end{column}
     	\begin{column}[T]{6cm} 
         	\begin{center}
            		\includegraphics[width=0.8\textwidth]{Radsatz}\source{Quelle: Falk2}
        		\end{center}
     \end{column}
 \end{columns}
}

\frame{\frametitle{Radsatzlager}
\framesubtitle{}
\begin{columns}[t] 
     \begin{column}[T]{6cm} 
     	\begin{itemize}
	\item Heute \"uberwiegend W\"alzlager
     		\item Zylindrische Lager:
		\begin{itemize}
		\item Vorteile bei der \"Ubertragung von Radsatzlasten
		\item Wenig bis keine Querf\"uhrung
		\end{itemize}
		\item Konische Lager:
		\begin{itemize}
		\item Reduzierte ertragbare Radsatzlasten
		\item Sehr gute Querf\"uhrung
		\end{itemize}
     	\end{itemize}
     \end{column}
     	\begin{column}[T]{6cm} 
         	\begin{center}
            		\includegraphics[width=0.8\textwidth]{RadsatzGleitlager}\source{Quelle: Ketamin}
        		\end{center}
     \end{column}
 \end{columns}
}

\frame{\frametitle{R\"ader}
\framesubtitle{}
\textbf{Unterscheidung}
\begin{itemize}
\item Konstruktionsprinzip:
	\begin{itemize}
		\item Einteilig
		\item Bereift
	\end{itemize}
\item Querschnitt:
	\begin{itemize}
		\item Gerade
		\item S-f\"ormig
		\item Konisch
		\item Wellenform
	\end{itemize}	
\end{itemize}
}

\offslide{Begrenzungen und Anschl\"age}

\offslide{Federcharakteristika}

\frame{\frametitle{Verbindung Drehgestell - Wagenkasten}
\framesubtitle{}
\begin{columns}[t] 
     \begin{column}[T]{6cm} 
     	\begin{itemize}
     		 \item Drehpfanne
		\begin{itemize}
		\item Flach
		\item Kugelig
		\end{itemize}
		\item Drehbar um Drehzapfen
		\item Meist \"Ubertragung der L\"angskr\"afte
		\item Evtl. zus\"atzlich Abst\"utzung auf Gleitplatten
		\item Weitere Verbindungen:
		\begin{itemize}
		\item Wankst\"utze
		\item Schlingerd\"ampfer
		\end{itemize}
     	\end{itemize}
     \end{column}
     	\begin{column}[T]{6cm} 
         	\begin{center}
            		\includegraphics[width=0.8\textwidth]{Drehpfanne}\source{Quelle: Manuel Schneider}
        		\end{center}
     \end{column}
 \end{columns}
}



\frame{\frametitle{Radsatzaufh\"angung}
\framesubtitle{}
\begin{columns}[t] 
     \begin{column}[T]{6cm} 
     	\begin{itemize}
     		\item \"Ublich: zweistufige Federung
		\begin{itemize}
		\item Prim\"arstufe: 
		\begin{itemize}
		\item Radsatz gegen Drehgestellrahmen
		\item Beschleunigung bis 100 g
		\end{itemize}
		\item Sekund\"arstufe: 
		\begin{itemize}
		\item Drehgestellrahmen gegen Fahrzeug
		\item Hohe Anforderungen an D\"ampfung
		\end{itemize}
		\end{itemize}
		\item<3> Bei G\"uterwagen auch einstufige Federung 
     	\end{itemize}
     \end{column}
     	\begin{column}[T]{6cm} 
         	\begin{center}
			\only<1>{
            		\includegraphics[width=0.8\textwidth]{SAGThameslink}\source{Quelle: Siemens Pressebild}}
			\only<2>{
            		\includegraphics[width=0.8\textwidth]{WattLinkage}\source{Quelle: Cdang/Tennen-Gas}}
			\only<3>{
            		\includegraphics[width=0.8\textwidth]{RadsatzGleitlager}\source{Quelle: Ketamin}}
        		\end{center}
     \end{column}
 \end{columns}
}

\frame{\frametitle{Verbindung Drehgestell - Wagenkasten}
\framesubtitle{}
\begin{columns}[t] 
     \begin{column}[T]{4cm} 
     	\begin{itemize}
     		\item Drehpfanne
\begin{itemize}
		\item Flach
		\item Kugelf\"ormig
		\end{itemize}
		\item Hochanlenkung
		\item Tiefanlenkung
     	\end{itemize}
     \end{column}
     	\begin{column}[T]{8cm} 
         	\begin{center}
            		\includegraphics[width=0.9\textwidth]{Tiefanlenkung}\source{Quelle: Christian Lindecke}
        		\end{center}
     \end{column}
 \end{columns}
}


\frame{\frametitle{Drehgestellrahmen}
\framesubtitle{}
\begin{columns}[t] 
     \begin{column}[T]{6cm} 
     	\begin{itemize}
		\item Form:
		\begin{itemize}
     		\item H-Form
		\item O-Form
		\end{itemize}
		\item Herstellung:
		\begin{itemize}
		\item Schwei{\ss}en
		\item Gie{\ss}en
		\end{itemize}
     	\end{itemize}
     \end{column}
     	\begin{column}[T]{6cm} 
         	\begin{center}
            		\includegraphics[width=0.8\textwidth]{Y25cast}\source{Quelle: A1AA1A} \\ \vspace{.5cm}  \includegraphics[width=0.8\textwidth]{Y25weld} \source{Quelle: A1AA1A}

        		\end{center}
     \end{column}
 \end{columns}
}
%\frame{\frametitle{Drehgestell-Anbauten}
%\framesubtitle{}
%\begin{itemize}
%\item Sandung
%\item Spurkranzschmierung
%\item Antennen
%\end{itemize}
%}


% Einf\"uhrung Zugdynamik
% !TEX root = SFV-14033_SFT1.tex
\section{Einf\"uhrung Zugdynamik}
\frame{\sectionpage}

\offslide{Einf\"uhrung Zugdynamik am Tafelbild}{Tafelbilder 1 - 5}





% Zugdynamik
% !TEX root = SFV-14033_SFT1.tex
\section{Zugdynamik}
\frame{\sectionpage}
%\subsection{Einf\"uhrung Zugdynamik}
%\frame{\subsectionpage}
%
%\offslide{Einf\"uhrung Zugdynamik am Tafelbild}{Tafelbilder 1 - 5}

\subsection{Kuppelsto{\ss}, Crash}
\frame{\subsectionpage}

\offslide{Reversibler Energieverzehr: L\"osungen, Wirkungsgrade}

\frame{\frametitle{Crash: Anforderungen der EN15227 \citep{dinen15227}}
\framesubtitle{}
\begin{center}
\begin{tabular}{|c|l|c|c|c|c|}
\hline
\multirow{2}{*}{Szenario} & \multirow{2}{*}{Hindernis} &\multicolumn{4}{c|} {Kollisionsgeschwindigkeit $v_{c}$}  \\ \cline{3-6}
& & C I & C II & C III & C IV\\ \hline
1 & Identische Zugeinheit & 36 & 25 & 25 & 15 \\ \hline
\multirow{2}{*}{2} & G\"uterwagen 80 t & 36 & - & 25 & - \\ \cline{2-6}
 & 129 t Regionalzug & - & - & 10 & - \\ \hline
 \multirow{2}{*}{3} & Deformierbar 15 t & $v_{lc} - 50$ & - & 25 & - \\ \cline{2-6}
 & Starr 3 t & - & - & - & 25 \\ \hline
\end{tabular}
\end{center}
	\begin{itemize}
		\item Zus\"atzlich: Anforderungen an Bahnr\"aumer
		\item \"Uberlebensraum und maximale Verz\"ogerungen m\"ussen eingehalten werden
		\item Nachweis \"uber Komponententests und validierte Modelle m\"oglich
	\end{itemize}
}
\offslide{Umsetzung Anforderungen EN 15227}

\subsection{Kraftschluss, Schlupf}
\frame{\subsectionpage}

\offslide{Kr\"afte am Rad}{Tafelbild 11}

\offslide{Physikalische Kraftschlusstheorie}{Tafelbild 12}

\offslide{Kraftschluss-Schlupf-Gesetz}{Tafelbild 13}

\offslide{Radschlupf: weitere Einfl\"usse}{Tafelbild 14}

\subsection{Fahrwiderstand, Zugkraft, Zugbremsung}
\frame{\subsectionpage}
\offslide{Sammlung Fahrwiderst\"ande am Tafelbild}

\offslide{Zugkraftdiagramm am Tafelbild}

\frame{\frametitle{Modelle f\"ur Zugdynamik}
\framesubtitle{}
\begin{itemize}
\item Massenpunktmodell
\begin{itemize}
	\item z.B. Einzelfahrzeuge, \"Uberschlagsrechnungen
\end{itemize}
\item Homogenes starres Massenbandmodell
\begin{itemize}
	\item z.B. Reisez\"uge
\end{itemize}
\item Inhomogenes starres Massenbandmodell
\begin{itemize}
	\item z.B. lange G\"uterz\"uge
\end{itemize}
\item Elastisches homogenes Massenbandmodell
\begin{itemize}
	\item z.B. Triebz\"uge
\end{itemize}
\item Elastisches inhomogenes Massenbandmodell
\begin{itemize}
	\item Allgemeines Modell
\end{itemize}
\end{itemize}
}


\offslide{Neigungskraft am Tafelbild}

\offslide{Zugbremsung am Tafelbild}




% Einf\"uhrung Spurf\"uhrung
% !TEX root = SFV-14033_SFT1.tex
\section{Einf\"uhrung Spurf\"uhrung}
\frame{\sectionpage}
\subsection{Spurweiten}

\frame{\frametitle{Spurweite \textit{track gauge}}
\framesubtitle{}
\begin{columns}[t] 
     \begin{column}[T]{5.9cm} 
     	\begin{itemize}
     		\item Spurweiten
		\begin{itemize}
		\item Begr\"undet aus wirtschaftlichen und milit\"arischen Motiven:
		\end{itemize}
		\begin{itemize}
		\item Regelspur: 1435 mm
		\item Breitspur \textit{wide gauge}
		\begin{itemize}
		\item Russische Spur: 1520 mm
		\item Indische Spur: 1676 mm
		\item Iberische Spur: 1668 mm
		\end{itemize}
		 \item Schmalspur \textit{narrow gauge}
		 \begin{itemize}
		\item Kapspur: 1067 mm
		\item Meterspur: 1000 mm
		\end{itemize}
		\end{itemize}
     	\end{itemize}
     \end{column}
     	\begin{column}[T]{6cm} 
         	\begin{definition}[Spurweite]
            		Die Spurweite ist der Abstand der Schienen zueinander, gemessen $(14{,}5 \pm 0{,}5) \, \mathrm{mm}$ unterhalb der Schienenoberkante \cite{tsiinf}. 
        		\end{definition}
		\begin{definition}[Spurweitentoleranz]
            		Abh\"angig von Netz und Strecke ist die Spurweite toleriert, \"ublich in Deutschland: $\left(1435^{+35}_{-5} \right) \mathrm{mm}$. 
        		\end{definition}
     \end{column}
 \end{columns}
 \begin{center}
 \includegraphics[width = 0.5\textwidth]{Trackgauge}\rotatebox{90}{{\tiny \color{gray} Quelle: ?}}
 \end{center}
 \note{
 \begin{itemize}
		\item Wie h\"angen r\"omische Soldaten und der Durchmesser der Space Shuttle-Tanks zusammen?
		\item Bezeichnung indische Loks.
		\end{itemize}
 }
}

\frame{\frametitle{Geografische Verteilung der Spurweiten}
\framesubtitle{}
\begin{center}
\includegraphics[width = 0.9\textwidth]{Railgaugeworld}\rotatebox{90}{{\tiny \color{gray} Quelle: ?}}
\end{center}
}

\offslide{Einf\"uhrung Spurf\"uhrung}{Tafelbilder 7 - 10}



\frame[allowframebreaks]{\frametitle{Literatur}
\framesubtitle{}
\nocite{wende}
\bibliographystyle{plainnat}
\bibliography{../../../../bib}
}

\end{document}
