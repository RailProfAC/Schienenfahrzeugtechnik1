% !TEX root = SFT1-Skript.tex
% !TEX root = SFT1-Folien.tex

\section{Pr\"aliminarien}

%\begin{frame}[plain]
%\begin{center}
%\vspace{2cm}
%\Huge Warum seid Ihr heute hier?\\ \vspace{.5cm}
%\pause \Large Die Zahlentheorie ist nützlich, weil man mit ihr promovieren kann.\\
%Edmund Landau
%\end{center}
%\end{frame}


\only<beamer>{

\frame{\frametitle{Prof. Dr. Raphael Pfaff}
\framesubtitle{Lehr- und Forschungsgebiet Schienenfahrzeugtechnik}
\begin{columns}[t] 
     \begin{column}[T]{7cm} 
     	\begin{itemize}
		\item[] \includegraphics[width=0.4cm]{Email} \hspace{.1cm} pfaff@fh-aachen.de
		\item[] \includegraphics[width=0.4cm]{Twitter} \hspace{.1cm} @RailProfAC
		\item[] \includegraphics[width=0.4cm]{Wordpress} \hspace{.1cm} www.raphaelpfaff.net
		\item[] Prezume: \texttt{\url{http://goo.gl/iq6lhh}}
		\vspace{1cm}
		\item Raum 02305
		\item Sprechstunde nach Vereinbarung
     	\end{itemize}
	
     \end{column}
     	\begin{column}[T]{5cm} 
         	\begin{center}
            		\includegraphics[width=0.8\textwidth]{Profilklein}
        		\end{center}
     \end{column}
 \end{columns}
}

\frame{\frametitle{Vorstellungsrunde}
\framesubtitle{}
\begin{itemize}
\item Wer bist Du?
\item Was erwartest Du von SFT1?
\item Was kann ich tun, damit SFT1 Dein Traummodul wird?
\item Was muss ich tun, damit Du SFT1 hasst?
\end{itemize}
}


\frame{\frametitle{Anforderungen ``First Cycle'' - Bachelor}
\framesubtitle{Anforderungen gem\"a{\ss} Dublin Descriptors}
\begin{columns}[t] 
     \begin{column}[T]{7cm} 
     	\begin{itemize}
     		\item Knowledge and understanding in a field of study
		\begin{itemize}
		\item Typically supported by textbooks
		\item Some aspects informed by knowledge on the forefront of the field of study
		\end{itemize}
		\item Apply knowledge and understanding indicating a professional approach
		\item Gather and interpret data to inform judgement
		\item Communicate information, ideas, problems and solutions
		\item Learning skills to undertake further study with high degree of autonomy
     	\end{itemize}
     \end{column}
     	\begin{column}[T]{5cm} 
         	\begin{center}
	\vspace{1cm}
            		\includegraphics[width=0.8\textwidth]{GraduationHat}
        		\end{center}
     \end{column}
 \end{columns}
}

\frame{\frametitle{Anforderungen ``Niveau 6'' - Bachelor}
\framesubtitle{Anforderungen gem\"a{\ss} Deutschem Qualifizierungsrahmen}
\begin{columns}[t] 
     \begin{column}[T]{7cm} 
     	\begin{itemize}
     		\item Breites und integriertes Wissen
		\begin{itemize}
		\item Wissenschaftliche Grundlagen
		\item Praktische Anwendungen 
		\end{itemize}
		\item Breites Spektrum an Methoden
		\begin{itemize}
		\item Neue L\"osungen erarbeiten und bewerten
		\end{itemize}
		\item Verantwortlich in Expertenteams arbeiten oder leiten
		\item Ziele f\"ur Lern- und Arbeitsprozesse definieren, reflektieren und bewerten
		\item Lern- und Arbeitsprozesse eigenst\"andig und nachhaltig gestalten
     	\end{itemize}
     \end{column}
     	\begin{column}[T]{5cm} 
         	\begin{center}
	\vspace{1cm}
            		\includegraphics[width=0.8\textwidth]{GraduationHat}
        		\end{center}
     \end{column}
 \end{columns}
}


\frame{\frametitle{Anforderungen BEng Schienenfahrzeugtechnik}
\framesubtitle{Was zeichnet einen Bachelor der Schienenfahrzeugtechnik aus?}
\begin{columns}[t] 
     \begin{column}[T]{6cm} 
     	\begin{itemize}
		\item Wissenschaftliches Arbeiten
		\begin{itemize}
		\item Nutzung Prim\"arliteratur und Normen
		\item Erstellung Seminararbeiten
		\end{itemize}
     		\item Selbstlernkompetenz
		\begin{itemize}
		\item Beispiel: Nutzung Lehrbuch statt Skript
		\end{itemize}
		\item Verfassung wissenschaftlicher und technischer Texte
		\item Fachvortrag zu Seminararbeit
     	\end{itemize}
     \end{column}
     	\begin{column}[T]{6cm} 
         	\begin{center}
	\vspace{1cm}
            		\includegraphics[width=0.8\textwidth]{GraduationHat}
        		\end{center}
     \end{column}
 \end{columns}
}
%
%\frame{\frametitle{Wie schaffen? - Growth Mindset!}
%\framesubtitle{Theorie der Psychologin Carol Dweck, Harvard \cite{mindset}}
%\begin{columns}[t] 
%     \begin{column}[T]{6cm} 
%     \textbf{Fixed Mindset:} 
%     	\begin{itemize}
%     		\item F\"ahigkeiten,
%		\item Intelligenz und
%		\item Talent 
%     	\end{itemize}
%	sind feste Pers\"onlichkeitsmerkmale und nicht \"anderbar. Daher das Bestreben, nicht dumm zu wirken.
%     \end{column}
%     	\begin{column}[T]{6cm} 
%	\textbf{Growth Mindset:} 
%         	\begin{itemize}
%     		\item F\"ahigkeiten,
%		\item Intelligenz und
%		\item Talent 
%     		\end{itemize}
%		k\"onnen durch Anstrengung, gute Lehre und Hartn\"ackigkeit entwickelt werden. Nicht jeder ist gleich, aber jeder kann sich weiterentwickeln.
%     \end{column}
% \end{columns}
%}
%
%\frame{\frametitle{Entwicklung im Growth Mindset}
%\framesubtitle{}
%\begin{center}
%\begin{tikzpicture}[scale = .9, minimum height = 20]
%\node (fixed)[minimum width = 70, draw=red!50!black, fill = red!20, rounded corners] at ( .6,8.75) {\textcolor{red!50!black}{Fixed mindset}};
%\node (growth)[minimum width = 70, draw=green!50!black, fill = green!20, rounded corners]  at ( 4.3,8.75){\textcolor{green!50!black}{Growth mindset}};
%
%\node (challenges)[text width = 320, draw=blue!30, fill = blue!20, align = left, minimum height = 24] at ( 0,7.5) {\textcolor{blue!40!black}{Herausforderungen}};
%\node (obstacles)[text width = 320, draw=blue!30, fill = blue!20, align = left, minimum height = 24] at ( 0,6.25) {\textcolor{blue!40!black}{Hindernisse}};
%\node (effort)[text width = 320, draw=blue!30, fill = blue!20, align = left, minimum height = 24] at ( 0,5) {\textcolor{blue!40!black}{Anstrengung}};
%\node (criticism)[text width = 320, draw=blue!30, fill = blue!20, align = left, minimum height = 24] at ( 0,3.75) {\textcolor{blue!40!black}{Kritik}};
%\node (success)[text width = 320, draw=blue!30, fill = blue!20, align = left, minimum height = 24] at ( 0,2.5) {\textcolor{blue!40!black}{Erfolg anderer}};
%
%\begin{scope}[draw = red!50!black]
%\uncover<2-7>{\node (avoid)[minimum width = 70, draw, fill = red!20, rounded corners] at ( .6,7.5) {\textcolor{red!50!black}{Vermeiden}};
%\path[draw, draw=red!50, line width = 5pt] (fixed) -- (avoid);};
%\uncover<3-7>{\node (giveup)[minimum width = 70, draw, fill = red!20, rounded corners] at ( .6,6.25) {\textcolor{red!50!black}{Aufgeben}};
%\path[draw, draw=red!50, line width = 5pt] (avoid) -- (giveup);};
%\uncover<4-7>{\node (useless)[minimum width = 70, draw, fill = red!20, rounded corners] at ( .6,5) {\textcolor{red!50!black}{Nutzlos}};
%\path[draw, draw=red!50, line width = 5pt] (giveup) -- (useless);};
%\uncover<5-7>{\node (ignore)[minimum width = 70, draw, fill = red!20, rounded corners] at ( .6,3.75) {\textcolor{red!50!black}{Ignorieren}};
%\path[draw, draw=red!50, line width = 5pt] (useless) -- (ignore);};
%\uncover<6-7>{\node (threat)[minimum width = 70, draw, fill = red!20, rounded corners] at ( .6,2.5) {\textcolor{red!50!black}{Bedrohung}};
%\path[draw, draw=red!50, line width = 5pt] (ignore) -- (threat);};
%\end{scope}
%
%\begin{scope}[draw = green!50!black]
%\uncover<2-7>{\node (embrace)[draw, minimum width = 70, fill = green!20, rounded corners] at ( 4.3,7.5) {\textcolor{green!50!black}{Nutzen}};
%\path[draw,  line width = 5pt] (growth) -- (embrace);};
%\uncover<3-7>{\node (persist)[draw, minimum width = 70,  fill = green!20, rounded corners] at (4.3,6.25) {\textcolor{green!50!black}{Weitermachen}};
%\path[draw,  line width = 10pt] (embrace) -- (persist);};
%\uncover<4-7>{\node (valuable)[draw, minimum width = 70,  fill = green!20, rounded corners] at ( 4.3,5) {\textcolor{green!50!black}{Wertvoll}};
%\path[draw,  line width = 15pt] (persist) -- (valuable);};
%\uncover<5-7>{\node (learn)[draw, minimum width = 70,  fill = green!20, rounded corners] at ( 4.3,3.75) {\textcolor{green!50!black}{Lernen}};
%\path[draw,  line width = 20pt] (valuable) -- (learn);};
%\uncover<6-7>{\node (inspiration)[draw, minimum width = 70,  fill = green!20, rounded corners] at (4.3,2.5) {\textcolor{green!50!black}{Inspiration}};
%\path[draw,  line width = 25pt] (learn) -- (inspiration);};
%\end{scope}
%
%\end{tikzpicture}
%\end{center}
%}
%
%\frame{\frametitle{Rolle des Lehrenden}
%\framesubtitle{}
%\begin{columns}[t] 
%     \begin{column}[T]{6cm} 
%     \vspace{1cm}
%     	\begin{quote}
%     		A teacher is never a giver of truth; he is a guide, a pointer to the truth that each student must find for himself.
%     	\end{quote}
%	\flushright Bruce Lee
%	
%     \end{column}
%     	\begin{column}[T]{6cm} 
%         	\begin{center}
%            		\includegraphics[width=0.8\textwidth]{Bruce}
%        		\end{center}
%     \end{column}
% \end{columns}
%}


\frame[allowframebreaks]{
\frametitle{Inhalt der Vorlesung}
\tableofcontents
}


%\frame{\frametitle{Vorlesungsinhalte}
%\framesubtitle{}
%\begin{center}
%\begin{tikzpicture}[scale = 0.95, small mindmap, concept color=gray!30]
%\node [concept] {Schienen-fahrzeug-technik}
%child [grow = 0]{node[concept] {Grundlagen}
%child [grow = -30] {node[concept] {Systematik}}
%child [grow = 30]{node[concept] {Normen und Regeln}}}
%child [grow = 60]{node[concept] {Anfor-derungen}
%child [grow = -20] {node[concept] {Requirements Engineering}}
%child [grow = 20]{node[concept] {Lebenszyklus}}}
%child [grow = 130]{node[concept] {Fahrzeuge}
%child [grow = 30] {node[concept] {G\"uterwagen}}
%child [grow = 150]{node[concept] {Personen-fahrzeuge}}
%child [grow = 190]{node[concept] {Triebfahr-zeuge}}}
%child [grow = 180]{node[concept] {Zugf\"order-technik}
%child [grow = 160] {node[concept] {Spurf\"uhrung}}
%child [grow = 200]{node[concept] {Rad-Schiene-Kontakt}}}
%child [grow = 235]{node[concept] {Fahrzeug-konstruk-tion}
%child [grow = 160] {node[concept] {Bauformen}}
%child [grow = 200]{node[concept] {Mechanischer Aufbau}}}
%child [grow = 300]{node[concept] {Laufwerk}
%child [grow = 20] {node[concept] {Drehgestell}}
%child [grow = 210]{node[concept] {Radsatz}}
%child [grow = -20]{node[concept] {Federn und D\"ampfer}}
%};
%\end{tikzpicture}
%\end{center}
%}
}
\frame{\frametitle{Lernziele}
\framesubtitle{}
\begin{itemize}
\item Studierende können, ausgehend von einer Kenntnis der Grundlagen des Systems Schienenfahrzeug, Anforderungen an Schienenfahrzeuge und ihre Subsysteme formulieren.
\item Studierende kennen die Grundlagen von Spurführung und Zugdynamik sowie von Fahrzeug- und Drehgestellkontruktion.
\item Studierende können sich selbstständig in ein Thema der Schienenfahrzeugtechnik einarbeiten und zu den Ergebnissen präsentieren.
\item Studierende können Mess- und Simulationsergebnisse in Berichtform dokumentieren. 
\end{itemize}
}


\offslide{Fehlt etwas?}{Was k\"onnt Ihr noch gebrauchen? z.B. f\"ur die Railway Challenge, euer Mobilit\"atsfenster, ...}


\frame{\frametitle{Themenplan, Kapitel aus \cite{ihme2016schienenfahrzeugtechnik}}
%\tiny
\href{https://link.springer.com/book/10.1007/978-3-658-24923-6}{Link zum Buch (Im VPN)}

\hspace{1cm}
\begin{tabular}{|l|l|}
\hline
Thema & Kapitel \\ \hline
Requirements Engineering & Lehrbrief +  Video\\ \hline
Zugdynamik \& Fahrwiderstand & 2 \\ \hline
Fahrzeugkonstruktion & 6  \\ \hline
Laufwerke  & 5, 6.6   \\ \hline
Einf\"uhrung Spurf\"uhrung & Lehrbrief +  Video\\ \hline
Kuppelsto{\ss}, Crash & Lehrbrief +  Video  \\ \hline
\end{tabular}
}

\frame{\frametitle{Selbststudium und semesterbegleitende Pr\"ufung}
\framesubtitle{}
\begin{itemize}
\item Konzepterstellung Parkbahn-Lokomotive
\begin{itemize}
	 \item Nach IMechE-Lastenheft + Zusatzanforderungen
	\item Ggf. enger Kostenrahmen
		\end{itemize}
\item Ablauf
\begin{enumerate}
	\item Anforderungsabdeckung / Grobkonzept (High-Level Design)
	\item Detailed Design (Standardisierter Antriebsstrang und Bremse)
	\begin{enumerate}[a)]
		\item Wagenkasten
		\item Drehgestell
		\end{enumerate}
	\item Nachweisf\"uhrung
	\begin{enumerate}[a)]
		\item Wagenkasten: Festigkeit
		\item Drehgestell: Entgleisungssicherheit
	\end{enumerate}
\end{enumerate}
\item Dokumentation durch technische Berichte
\item Gewichtung Berichte: in Summe 100\% der Modulnote
\end{itemize}
}



