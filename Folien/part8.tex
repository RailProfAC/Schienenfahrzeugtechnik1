% !TEX root = SFV-14033_SFT1.tex
\section{Einf\"uhrung Spurf\"uhrung}
\frame{\sectionpage}
\subsection{Spurweiten}

\frame{\frametitle{Spurweite \textit{track gauge}}
\framesubtitle{}
\begin{columns}[t] 
     \begin{column}[T]{5.9cm} 
     	\begin{itemize}
     		\item Spurweiten
		\begin{itemize}
		\item Begr\"undet aus wirtschaftlichen und milit\"arischen Motiven:
		\end{itemize}
		\begin{itemize}
		\item Regelspur: 1435 mm
		\item Breitspur \textit{wide gauge}
		\begin{itemize}
		\item Russische Spur: 1520 mm
		\item Indische Spur: 1676 mm
		\item Iberische Spur: 1668 mm
		\end{itemize}
		 \item Schmalspur \textit{narrow gauge}
		 \begin{itemize}
		\item Kapspur: 1067 mm
		\item Meterspur: 1000 mm
		\end{itemize}
		\end{itemize}
     	\end{itemize}
     \end{column}
     	\begin{column}[T]{6cm} 
         	\begin{definition}[Spurweite]
            		Die Spurweite ist der Abstand der Schienen zueinander, gemessen $(14{,}5 \pm 0{,}5) \, \mathrm{mm}$ unterhalb der Schienenoberkante \cite{tsiinf}. 
        		\end{definition}
		\begin{definition}[Spurweitentoleranz]
            		Abh\"angig von Netz und Strecke ist die Spurweite toleriert, \"ublich in Deutschland: $\left(1435^{+35}_{-5} \right) \mathrm{mm}$. 
        		\end{definition}
     \end{column}
 \end{columns}
 \begin{center}
 \includegraphics[width = 0.5\textwidth]{Trackgauge}\rotatebox{90}{{\tiny \color{gray} Quelle: ?}}
 \end{center}
 \note{
 \begin{itemize}
		\item Wie h\"angen r\"omische Soldaten und der Durchmesser der Space Shuttle-Tanks zusammen?
		\item Bezeichnung indische Loks.
		\end{itemize}
 }
}

\frame{\frametitle{Geografische Verteilung der Spurweiten}
\framesubtitle{}
\begin{center}
\includegraphics[width = 0.9\textwidth]{Railgaugeworld}\rotatebox{90}{{\tiny \color{gray} Quelle: ?}}
\end{center}
}

\offslide{Einf\"uhrung Spurf\"uhrung}{Tafelbilder 7 - 10}
