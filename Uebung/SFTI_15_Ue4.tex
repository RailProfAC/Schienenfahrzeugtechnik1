\documentclass[10pt,a4paper,headsepline,smallheadings]{scrartcl}
\usepackage[utf8]{inputenc}
\usepackage[T1]{fontenc}
\usepackage[ngerman]{babel}
\usepackage{amsmath}
\usepackage{amsthm}
\usepackage{amssymb}
\usepackage{amsfonts}
\usepackage[scaled]{helvet}
\usepackage{amssymb}
\usepackage{multirow}
\usepackage{textcomp}
\usepackage{graphicx}
\usepackage{paralist}
\usepackage{textcomp}
\usepackage{pdflscape} 
\usepackage{marvosym}
\usepackage{float}
\usepackage{siunitx}
\usepackage[siunitx,european,cuteinductors,smartlabels]{circuitikz}
\usepackage{fancyhdr}
\usepackage{pgfplots}
\usepackage{sansmath}
\usetikzlibrary{shapes.geometric}
\usepackage{wasysym}

\usetikzlibrary{calc}


\theoremstyle{definition}
\newtheorem{aufgabe}{Aufgabe}


\renewcommand*\familydefault{\sfdefault}
\renewcommand{\arraystretch}{1.1}


\KOMAoptions{parskip=half,DIV=15,fontsize=11pt}
\unitlength1cm
\titlehead{

\begin{center}\begin{tabular}{p{10cm}p{6.8cm}}
\textbf{FH Aachen} & \textbf{FB Maschinenbau und Mechatronik}\\[0.5cm]
\textbf{Modul 86111} &  Prof. Dr. Raphael Pfaff\\
Schienenfahrzeugtechnik I& Sommersemester 2015\\
\end{tabular}

\end{center}
\begin{picture}(0,0)(0,0)\put(17,-23){\includegraphics[height=5cm]{fh_logo}}\end{picture}
}
\newif\ifuelsg %als slides
%\uelsgtrue
\uelsgfalse
\newif\ifnotuelsg
\ifuelsg\notuelsgfalse\else\notuelsgtrue\fi


%\titlehead{83105 \hfill Mess-Steuerungs- und Regelungstechnik \hfill Prof. Manfred Enning}
\title{Schienenfahrzeugtechnik I  -- \"Ubung 4}
\date{}
\makeatletter
\let\Title\@title
\let\Author\@author
\makeatother
\pagestyle{fancy}
\fancyhead[LE, LO]{Prof. Dr. Raphael Pfaff}
\fancyhead[RE, RO]{\Title}

\begin{document}
\thispagestyle{empty}
\maketitle
\vspace{-2cm}
% \hyphenation{Abwei-chungen}

\section*{Bremskurven}
\begin{aufgabe}[Bremsarten]
Ein zweistufiges Bremsmodell nutzt folgende Stufen, um einen Bremsprozess mit Füllzeit $t_f$ und Verzögerung $\bar{a}$ zu modellieren:
\begin{itemize}
	\item Konstantfahrt für eine Dauer von $\frac12 t_f$
	\item Konstante Verzögerung $\bar{a}$ von $\frac12 t_f$ bis zum Stillstand des Fahrzeugs
\end{itemize}
\begin{enumerate}[a)]
\item Skizzieren Sie die Bremskurven im v-s-Diagramm, wobei der Unterschied zwischen den Bremsarten herausgestellt werden soll.
\item Welchen Einfluss hat eine l\"angere F\"ullzeit auf
\begin{itemize}
	\item die zul\"assige H\"ochstgeschwindigkeit im Vorsignal-Hauptsignal-System?
	\item die L\"angsdruckkr\"afte w\"ahrend einer Bremsung?
\end{itemize}
\end{enumerate}
\end{aufgabe}
\vspace{.5cm}
\begin{aufgabe}[L\"angsdruckkr\"afte]
In einem Zugverband, der ansonsten frei von L\"angszug- und -druckkr\"aften ist, befinden sich zwei Gelenktragwagen der Bauart Sggrss (L\"uP = 27 m). Die verbauten Steuerventile der Wagen liegen ung\"unstig verteilt im Rahmen der Toleranzen gem\"a{\ss} UIC 540 in Bremsstellungen G bzw. P , sodass im vorderen Wagen die k\"urzeste und im hinteren Wagen die l\"angste F\"ullzeit erreicht wird. Bestimmen Sie die auftretenden L\"angsdruckkr\"afte unter folgenden Annahmen:
\begin{itemize}
	\item Durchschlagsgeschwindigkeit $v_{SB} = 250 \frac{\mathrm{m}}{\mathrm{m}}$
	\item Linearer Aufbau der Bremskraft von $0$ auf $F_{B, max}$ innerhalb der F\"ullzeit $t_{f}$
	\item Masse der Wagen $m_{W} = 60 \mathrm{t}$, rotierende Masse $m_{D} = 2{,}5 \mathrm{t}$
	\item Bremskraft am Radumfang $F_{B} = 60 \mathrm{kN}$
\end{itemize}
\end{aufgabe}

\section*{Fahrwiderstand}
\begin{aufgabe}[Fahrwiderstand nach Strahl und Sauthoff]
\begin{enumerate}[a)]
\item Berechnen Sie die ben\"otigte Energie f\"ur je $s = 100 \mathrm{km}$ Streckenfahrt mit $v_{max}$:
\begin{itemize}
		\item Gemischter G\"uterzug, $v_{max} = 80 \frac{\mathrm{km}}{\mathrm{h}}$, $m_{W} = 4000 \mathrm{t}$
		\begin{itemize}
		\item Widerstandsgleichung nach Strahl:
		\begin{equation}
		\label{Eq:Strahl}
			f_{WW} = 1{,}6 \permil +  5{,}7 \permil \left( \frac{v}{100 \frac{\mathrm{km}}{\mathrm{h}}} \right)^2
		\end{equation}
		\end{itemize}
		\item Reisezug, $v_{max} = 160 \frac{\mathrm{km}}{\mathrm{h}}$, $m_{W} = 350 \mathrm{t}$, $n_{W} = 7$
 		\begin{itemize}
		\item Widerstandsgleichung nach Sauthoff:
		\begin{equation}
		\label{Eq:Sauthoff}
			f_{WW} = 1{,}6 \permil + 0{,}25 \permil \left( \frac{v}{100 \frac{\mathrm{km}}{\mathrm{h}}} \right) + 
			\frac{683 \mathrm{N}(2{,}7 + n_{W}) }{m_{W} g} \left( \frac{v+ 12 \frac{\mathrm{km}}{\mathrm{h}}}{100 \frac{\mathrm{km}}{\mathrm{h}}} \right)^2
		\end{equation}
		\end{itemize}
		\end{itemize}
\item Berechnen Sie die ben\"otigte Energie f\"ur das Beschleunigen der Z\"uge auf $v_{max}$ unter Ber\"ucksichtigung des Fahrwiderstands gem\"a{\ss} der Gleichungen \eqref{Eq:Strahl} bzw. \eqref{Eq:Sauthoff} sowie einer konstanten Beschleunigung von
\begin{itemize}
		\item $a = 0{,}1 \frac{\mathrm{m}}{\mathrm{s}^2}$ f\"ur den G\"uterzug
		\item $a = 0{,}3 \frac{\mathrm{m}}{\mathrm{s}^2}$ f\"ur den Personenzug
		\end{itemize}
\end{enumerate}
Der Widerstand des Triebfahrzeugs ist zu vernachl\"assigen.
\end{aufgabe}

%\begin{aufgabe}[Massenband/Massenpunktmodell]
%Ein elfteiliger Triebzug f\"ahrt auf einer parabelf\"ormigen Strecke, die der Vorschrif
% 
%\end{aufgabe}

\end{document}