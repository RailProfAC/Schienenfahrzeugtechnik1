\documentclass[10pt,a4paper,headsepline,smallheadings]{scrartcl}
\usepackage[utf8]{inputenc}
\usepackage[T1]{fontenc}
\usepackage[ngerman]{babel}
\usepackage{amsmath}
\usepackage{amsthm}
\usepackage{amssymb}
\usepackage{amsfonts}
\usepackage[scaled]{helvet}
\usepackage{amssymb}
\usepackage{multirow}
\usepackage{textcomp}
\usepackage{graphicx}
\usepackage{paralist}
\usepackage{textcomp}
\usepackage{pdflscape} 
\usepackage{marvosym}
\usepackage{float}
\usepackage{siunitx}
\usepackage[siunitx,european,cuteinductors,smartlabels]{circuitikz}
\usepackage{fancyhdr}
\usepackage{pgfplots}
\usepackage{sansmath}
\usetikzlibrary{shapes.geometric}
\usepackage{wasysym}

\usetikzlibrary{calc}


\theoremstyle{definition}
\newtheorem{aufgabe}{Aufgabe}


\renewcommand*\familydefault{\sfdefault}
\renewcommand{\arraystretch}{1.1}


\KOMAoptions{parskip=half,DIV=15,fontsize=11pt}
\unitlength1cm
\titlehead{

\begin{center}\begin{tabular}{p{10cm}p{6.8cm}}
\textbf{FH Aachen} & \textbf{FB Maschinenbau und Mechatronik}\\[0.5cm]
\textbf{Modul 84111} &  Prof. Dr. Raphael Pfaff\\
Schienenfahrzeugtechnik I& Sommersemester 2015\\
\end{tabular}

\end{center}
\begin{picture}(0,0)(0,0)\put(17,-23){\includegraphics[height=5cm]{fh_logo}}\end{picture}
}
\newif\ifuelsg %als slides
%\uelsgtrue
\uelsgfalse
\newif\ifnotuelsg
\ifuelsg\notuelsgfalse\else\notuelsgtrue\fi


%\titlehead{83105 \hfill Mess-Steuerungs- und Regelungstechnik \hfill Prof. Manfred Enning}
\title{Schienenfahrzeugtechnik I  -- \"Ubung 6}
\date{}
\makeatletter
\let\Title\@title
\let\Author\@author
\makeatother
\pagestyle{fancy}
\fancyhead[LE, LO]{Prof. Dr. Raphael Pfaff}
\fancyhead[RE, RO]{\Title}

\begin{document}
\thispagestyle{empty}
\maketitle
\vspace{-2cm}
% \hyphenation{Abwei-chungen}

\section*{L\"angsdynamik}

\begin{aufgabe}[Radsatzwellen]
Sizzieren Sie die Verl\"aufe der Momente $M_{x}$ und $M_{z}$ entlang der Radsatzwelle f\"ur folgende F\"alle:
\begin{enumerate}
\item Aussengelagert:
	\begin{enumerate}[a)]
	\item Klotzbremse einseitig auf das Rad wirkend
	\item Klotzbremse beidseitig auf das Rad wirkend
	\item Eine Wellenbremsscheibe
	\item Zwei Wellenbremsscheiben
	\item Zwei Radbremsscheiben
	\end{enumerate}
\item Innengelagert:
	\begin{enumerate}[a)]
	\item Klotzbremse einseitig auf das Rad wirkend
	\item Klotzbremse beidseitig auf das Rad wirkend
	\item Eine Wellenbremsscheibe
	\item Zwei Wellenbremsscheiben
	\item Zwei Radbremsscheiben
	\end{enumerate}
\end{enumerate}

\end{aufgabe}
\newpage
\begin{aufgabe}[Rads\"atze/W\"armeeintrag]
Ein bereiftes Rad mit Laufkreisdurchmesser $d = 920\; \mathrm{mm}$ wird in einem Gef\"alle mit einer Klotzbremse dauergebremst, um die Geschwindigkeit konstant zu halten. 

\begin{itemize}
	\item Spezifische W\"armekapazit\"at Stahl: $c = 477 \, \frac{\mathrm{J}}{\mathrm{kg}\, \mathrm{K}}$
	\item Geschwindigkeit: $v = 70\, \frac{\mathrm{km}}{\mathrm{h}}$
	\item Radsatzlast: $ 2 Q = 22{,5} \, \mathrm{t}$
	\item Streckenneigung: $i_{k} = \{2, 4\} \%$
\end{itemize}
\begin{enumerate}[a)]
\item Bestimmen Sie den Leistungseintrag der Klotzbremse w\"ahrend der Beharrungsbremsung in den beiden angegeben Streckenneigung unter folgenden Annahmen:
	\begin{itemize}
		\item Es findet kein Transfer von abzubremsenden Massen statt, d.h. jedes Rad bremst sich selbst
		\item Die Verbundsohle nimmt 10 \% des Leistungseintrags auf
	\end{itemize}
\item Bestimmen Sie die Temperaturentwicklung im Radreifen unter folgenden Annahmen:
	\begin{itemize}
		\item Keine W\"armeleitung in den Radsteg
		\item Radreifendicke: $d = 90 \, \mathrm{mm}$ 
		\item Radbreite: $b = 140 \, \mathrm{mm}$
	\end{itemize}
\end{enumerate}
\end{aufgabe}


\end{document}