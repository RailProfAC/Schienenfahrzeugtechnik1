\documentclass[11pt,a4paper,headsepline]{scrartcl}
\usepackage[utf8]{inputenc}
\usepackage[german]{babel}
\usepackage{amsmath}
\usepackage{amsfonts}
\usepackage{amssymb}
\usepackage{tikz}
\usepackage[left=2cm,right=2cm,top=2cm,bottom=2cm]{geometry}
\usepackage[utf8]{inputenc}
\usepackage{graphicx}
\usepackage{pgfplots}
\usepackage{fancyhdr}
\usepackage[scaled]{helvet}

\usetikzlibrary{calc}
\newtheorem{aufgabe}{Aufgabe}


\renewcommand*\familydefault{\sfdefault}
\renewcommand{\arraystretch}{1.1}

\title{\"Ubung Festigkeitsanforderungen und -nachweis}
\date{}
\makeatletter
\let\Title\@title
\let\Author\@author
\makeatother
\pagestyle{fancy}
\fancyhead[L]{Prof. Dr. Raphael Pfaff}
\fancyhead[R]{84111: Schienenfahrzeugtechnik}
\fancyfoot[L]{Datei: \jobname}
\fancyfoot[R]{Datum: \today
\begin{picture}(0,0)(0,0)\put(.5,0){\includegraphics[height=5cm]{fh_logo}}\end{picture}}

\begin{document}
\maketitle
\thispagestyle{fancy}
\pagestyle{fancy}
\vspace{-2cm}

\begin{aufgabe}[Festigkeitsanforderungen] 
\label{Ta:Festigkeit}
Als Absolvent(in) der Schienenfahrzeugtechnik bei einem Zulieferunternehmen ist es Ihre erste Aufgabe, einen Sensor (z.B. ein Dopplerradar) mit einer Masse $m = 50 \, \mathrm{kg}$ in einem Abstand von $x_{A} = 200 \, \mathrm{mm}$ zur Anschraubfl\"ache im Bereich der Pufferbohle zu befestigen. Aufgrund der beengten Einbaulage w\"ahlen Sie eine Kragarmkonstruktion.
\begin{figure}[htbp]
\begin{center}
\begin{tikzpicture}
\draw[fill = gray!30] (-2,-2) -- (-2,2) -- (2,2) -- (2,-2) -- (-2,-2);
\filldraw (0,0) circle (0.1) node[above = 1]  {$m$};
\draw[ultra thick] (2,0) -- (7,0);
\draw[ultra thick] (7,-2) -- (7,2);
\foreach \y in {-1.8, -1.6, ..., 2.0}
	{\draw[thick] (7, \y) -- (7.2, \y-.2);
	}
\draw[<->, thick] (0,-.8) -- (7,-.8) node[pos = 0.5, above] {$x_{A}$};
\draw[->, thick] (-3,-3) -- (-1,-3) node[pos = 0.5, above] {$x$};
\draw[->, thick] (-3,-3) -- (-3,-1) node[pos = 0.5, right] {$z$};
\end{tikzpicture}				
\caption{Darstellung des Aufbaus}
\label{Fig:Festigkeit}
\end{center}
\end{figure}

\begin{enumerate}
		\item Was sind die Nachteile einer solchen Kragarmkonstruktion?
		\item Welche alternativen Konstruktionen sollten gew\"ahlt werden?
		\item Bestimmen Sie die Anforderungen an den Festigkeitsnachweise gem\"a{\ss} EN 12663-1 und EN 12663-2 f\"ur folgenden F\"alle:
		\begin{enumerate}
		\item Statischer Nachweis, Personenfahrzeug
		\item Nachweis auf Dauerfestigkeit, Personenfahrzeug
		\item Statischer Nachweis, G\"uterwagen
		\item Nachweis auf Dauerfestigkeit, G\"uterwagen
		\end{enumerate}
\end{enumerate}
Gehen Sie dabei von einem Stahl S 355 aus.
\end{aufgabe}

\begin{aufgabe}[Festigkeitsanforderungen] 
\label{Ta:Nachweis}
F\"uhren Sie den Festigkeitsnachweis f\"ur die oben bestimmten Lastf\"alle f\"ur eine Kragarm aus einem Rundstahl $d = 25 \, \mathrm{mm}$. Als Material nehmen Sie  S355 mit einer Streckgrenze von $355\, \frac{\mathrm{N}}{\mathrm{mm}^2}$ sowie einer dauerfest ertragbaren Biegespannung von $235\, \frac{\mathrm{N}}{\mathrm{mm}^2}$ an.
\end{aufgabe}
\end{document}