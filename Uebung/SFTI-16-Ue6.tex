\documentclass[11pt,a4paper,headsepline]{scrartcl}
\usepackage[utf8]{inputenc}
\usepackage[german]{babel}
\usepackage{amsmath}
\usepackage{amsfonts}
\usepackage{amssymb}
\usepackage{tikz}
\usepackage[left=2cm,right=2cm,top=2cm,bottom=2cm]{geometry}
\usepackage[utf8]{inputenc}
\usepackage{graphicx}
\usepackage{pgfplots}
\usepackage{fancyhdr}
\usepackage[scaled]{helvet}
\usepackage{paralist}
\usepackage{pdfpages}

\usetikzlibrary{calc}
\newtheorem{aufgabe}{Aufgabe}

\renewcommand*\familydefault{\sfdefault}
\renewcommand{\arraystretch}{1.1}

\title{\"Ubung Rads\"atze}
\date{}
\makeatletter
\let\Title\@title
\let\Author\@author
\makeatother
\pagestyle{fancy}
\fancyhead[L]{Prof. Dr. Raphael Pfaff}
\fancyhead[R]{84111: Schienenfahrzeugtechnik}
\fancyfoot[L]{Datei: \jobname}
\fancyfoot[R]{Datum: \today \vspace{.1cm}
\begin{picture}(0,0)(0,0)\put(.5,1){\includegraphics[height=5cm]{fh_logo}}\end{picture}}

\begin{document}
\maketitle
\thispagestyle{fancy}
\pagestyle{fancy}
\vspace{-2cm}
\begin{aufgabe}[Radsatzwellen]
Sizzieren Sie die Verl\"aufe der Momente $M_{x}$, $M_{y}$ und $M_{z}$ entlang der Radsatzwelle f\"ur folgende F\"alle:
	\begin{enumerate}[a)]
	\item Klotzbremse einseitig auf das Rad wirkend
	\item Klotzbremse beidseitig auf das Rad wirkend
	\item Eine Wellenbremsscheibe
	\item Zwei Wellenbremsscheiben
	\item Zwei Radbremsscheiben
	\end{enumerate}
\end{aufgabe}
\vspace{.5cm}
\begin{aufgabe}[Rads\"atze/W\"armeeintrag]
Ein bereiftes Rad mit Laufkreisdurchmesser $d = 920\; \mathrm{mm}$ wird in einem Gef\"alle mit einer Klotzbremse dauergebremst, um die Geschwindigkeit konstant zu halten. 

\begin{itemize}
	\item Spezifische W\"armekapazit\"at Stahl: $c = 477 \, \frac{\mathrm{J}}{\mathrm{kg}\, \mathrm{K}}$
	\item Geschwindigkeit: $v = 70\, \frac{\mathrm{km}}{\mathrm{h}}$
	\item Radsatzlast: $ 2 Q = 22{,5} \, \mathrm{t}$
	\item Streckenneigung: $i_{k} = \{2, 4\} \%$
\end{itemize}
\begin{enumerate}[a)]
\item Bestimmen Sie den Leistungseintrag der Klotzbremse w\"ahrend der Beharrungsbremsung in den beiden angegeben Streckenneigung unter folgenden Annahmen:
	\begin{itemize}
		\item Es findet kein Transfer von abzubremsenden Massen statt, d.h. jedes Rad bremst sich selbst
		\item Die Verbundsohle nimmt 10 \% des Leistungseintrags auf
	\end{itemize}
\item Bestimmen Sie die Temperaturentwicklung im Radreifen unter folgenden Annahmen:
	\begin{itemize}
		\item Keine W\"armeleitung in den Radsteg
		\item Radreifendicke: $d = 90 \, \mathrm{mm}$ 
		\item Radbreite: $b = 140 \, \mathrm{mm}$
	\end{itemize}
\end{enumerate}
\end{aufgabe}

\end{document}