\documentclass[10pt,a4paper,headsepline,smallheadings]{scrartcl}
\usepackage[utf8]{inputenc}
\usepackage[T1]{fontenc}
\usepackage[ngerman]{babel}
\usepackage{amsmath}
\usepackage{amsthm}
\usepackage{amssymb}
\usepackage{amsfonts}
\usepackage[scaled]{helvet}
\usepackage{amssymb}
\usepackage{multirow}
\usepackage{textcomp}
\usepackage{graphicx}
\usepackage{paralist}
\usepackage{textcomp}
\usepackage{pdflscape} 
\usepackage{marvosym}
\usepackage{float}
\usepackage{siunitx}
\usepackage[siunitx,european,cuteinductors,smartlabels]{circuitikz}
\usepackage{fancyhdr}
\usepackage{pgfplots}
\usepackage{sansmath}
\usetikzlibrary{shapes.geometric}

\usetikzlibrary{calc}


\theoremstyle{definition}
\newtheorem{aufgabe}{Aufgabe}


\renewcommand*\familydefault{\sfdefault}
\renewcommand{\arraystretch}{1.1}


\KOMAoptions{parskip=half,DIV=15,fontsize=11pt}
\unitlength1cm
\titlehead{

\begin{center}\begin{tabular}{p{10cm}p{6.8cm}}
\textbf{FH Aachen} & \textbf{FB Maschinenbau und Mechatronik}\\[0.5cm]
\textbf{Modul 86111} &  Prof. Dr. Raphael Pfaff\\
Schienenfahrzeugtechnik I& Sommersemester 2015\\
\end{tabular}

\end{center}
\begin{picture}(0,0)(0,0)\put(17,-23){\includegraphics[height=5cm]{fh_logo}}\end{picture}
}
\newif\ifuelsg %als slides
%\uelsgtrue
\uelsgfalse
\newif\ifnotuelsg
\ifuelsg\notuelsgfalse\else\notuelsgtrue\fi
\graphicspath{
{../Bilder/Uebungen/}
{../Bilder/Wirkungsplan/}
}

%\titlehead{83105 \hfill Mess-Steuerungs- und Regelungstechnik \hfill Prof. Manfred Enning}
\title{Schienenfahrzeugtechnik I  -- \"Ubung 2}
\date{}
\makeatletter
\let\Title\@title
\let\Author\@author
\makeatother
\pagestyle{fancy}
\fancyhead[LE, LO]{Prof. Dr. Raphael Pfaff}
\fancyhead[RE, RO]{\Title}

\begin{document}
\thispagestyle{empty}
\maketitle
\vspace{-2cm}
% \hyphenation{Abwei-chungen}

\section*{Einf\"uhrung in die Spurf\"uhrung}
\begin{aufgabe}[Erarbeitung Klingel'sche Formel] 
Bearbeiten Sie folgende Aufgabe in Kleingruppen (2-3 Studierende).

Betrachten Sie einen Einzelradsatz mit konischem Radprofil, zun\"achst in Querrichtung verschoben:
\begin{center}
	\begin{tikzpicture}[scale = 0.8]
	\draw[ultra thick, draw = red!70!black] (-2,-.4) -- (11,-.4){};
	\draw[ultra thick, draw = red!70!black] (-2,3.1) -- (11,3.1){};
	\draw[<->,thick] (10.5,-.4) -- (10.5,3.1);
	\node[right] at (10.5,1.35) {$2b$};
	\path[fill = blue!30, opacity = 0.5] (-1,-1) -- (4,-1) -- (4,4) --(-1,4) -- (-1,-1);
	\draw[thick, fill = gray!30] (0,3) -- (3,3) -- (2.8,3.5) -- (0.2,3.5) -- (0,3) {}; 
	\draw[thick, fill = gray!30] (1.3,0) -- (1.7,0) -- (1.7,3) -- (1.3,3) -- (1.3,0) {}; 
	\draw[thick, fill = gray!30] (0,0) -- (3,0) -- (2.8,-0.5) -- (0.2,-0.5) -- (0,0) {}; 
	\draw[->,thick] (-3,1.5) -- (-1.5,1.5);
	\node[above] at (-2.25,1.5) {$v$};
	\path[fill = green!30, opacity = 0.5] (5,-1) -- (10,-1) -- (10,4) --(5,4) -- (5,-1);
	\draw[thick, fill = gray!30, rotate around={-4:(7.5,1.5)}] (6,3) -- (9,3) -- (8.8,3.5) -- (6.2,3.5) -- (6,3) {}; 
	\draw[thick, fill = gray!30, rotate around={-4:(7.5,1.5)}] (7.3,0) -- (7.7,0) -- (7.7,3) -- (7.3,3) -- (7.3,0) {}; 
	\draw[thick, fill = gray!30, rotate around={-4:(7.5,1.5)}] (6,0) -- (9,0) -- (8.8,-0.5) -- (6.2,-0.5) -- (6,0) {}; 
	\draw[draw = blue!80, fill = blue!50] (7.63,0.5) -- (7.71,1.5) -- (7.71,0.5);
	\node[right] at (7.71,0.7) {\color{blue!80!black} $\varphi$};
	\draw[->,thick] (3.75,1.5) -- (5.25,1.5);
	\node[above] at (4.5,1.5) {$x$};
	\draw[->,thick] (3.75,1.5) -- (3.75,3);
	\node[right] at (3.75,2.25) {$y$};
	\draw[->,thick] (5.25,1.6) arc [->,start angle=0, end angle=90, radius=1.4];
	\node[right] at (5.1,2.25) {$\varphi$};
	\end{tikzpicture}
\end{center}

\begin{enumerate}
	\item Bestimmen Sie die Quergleitgeschwindigkeit abh\"angig von $v$ und $\varphi$.
	\begin{itemize}
		\item Vereinfachen Sie so, dass die Abh\"angigkeit von $\varphi$ linear ist.
		\end{itemize}
	\item Differenzieren Sie, um die Querbeschleunigung zu erhalten. Annahme: $v$ konstant.
	\item Bestimmen Sie die Winkelgeschwindigkeit $\dot{\varphi}$ des Radsatzes um die Hochachse abh\"angig von
	\begin{itemize}
		\item Rollradiendifferenz $\triangle r$,
		\item halbem Radstand $b$ sowie 
		\item Winkelgeschwindigkeit des Radsatzes $\omega$.
	\end{itemize}
	\item Vereinfachen Sie f\"ur kegelf\"ormige Rads\"atze $\triangle r$.
	\item Leiten Sie die homogene lineare Differenzialgeichung der Bewegung in $y$-Richtung aus den oben gefundenen Beziehungen her.
	\begin{itemize}
		\item Welche Eigenschaften hat diese Differenzialgleichung?
		\item Welche (wichtigen) Aspekte haben Sie vernachl\"assigt?
	\end{itemize}
	\item Bestimmen Sie Eigenkreisfrequenz und die Wellenl\"ange der Bewegung.
	\begin{itemize}
		\item Welche Beobachtungen k\"onnen Sie machen?
	\end{itemize}
\end{enumerate}


\end{aufgabe}

\end{document}