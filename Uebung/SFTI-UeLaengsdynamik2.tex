\documentclass[11pt,a4paper,headsepline]{scrartcl}
\usepackage[utf8]{inputenc}
\usepackage[german]{babel}
\usepackage{amsmath}
\usepackage{amsfonts}
\usepackage{amssymb}
\usepackage{tikz}
\usepackage[left=2cm,right=2cm,top=2cm,bottom=2cm]{geometry}
\usepackage[utf8]{inputenc}
\usepackage{graphicx}
\usepackage{pgfplots}
\usepackage{fancyhdr}
\usepackage[scaled]{helvet}
\usepackage{paralist}

\usetikzlibrary{calc}
\newtheorem{aufgabe}{Aufgabe}


\renewcommand*\familydefault{\sfdefault}
\renewcommand{\arraystretch}{1.1}

\title{\"Ubung L\"angsdynamik II}
\date{}
\makeatletter
\let\Title\@title
\let\Author\@author
\makeatother
\pagestyle{fancy}
\fancyhead[L]{Prof. Dr. Raphael Pfaff}
\fancyhead[R]{84111: Schienenfahrzeugtechnik}
\fancyfoot[L]{Datei: \jobname}
\fancyfoot[R]{Datum: \today \vspace{.1cm}
\begin{picture}(0,0)(0,0)\put(.5,1){\includegraphics[height=5cm]{fh_logo}}\end{picture}}

\begin{document}
\maketitle
\thispagestyle{fancy}
\pagestyle{fancy}
\vspace{-2cm}

\begin{aufgabe}[Massenband/Massenpunktmodell]
Ein siebenteiliger Triebzug ($m_{w} = 50 \, \mathrm{t}$, $l_{w} = 25 \, \mathrm{m}$) f\"ahrt auf einer Strecke, die der Vorschrift 
 \begin{equation*}
h(x) = 
\begin{cases}
 0, \quad x < 5000 \\
 - 100 \cos \frac{x-5000}{5000} + 100, \quad x \geq 5000
\end{cases}
\end{equation*}
entspricht. Hierbei wird die Position der Zugspitze $x$ in $\mathrm{m}$ gemessen.
\begin{enumerate}[a)]
\item Bestimmen Sie die maximale Streckenneigung $i_{max}$ der Strecke.
\item Bestimmen Sie den Punkt, an dem $E_{pot} > 0$ gilt im Massenband- bzw. Massenpunktmodell.
\item Bestimmen Sie f\"ur $x = 7000$ die Neigungswiderstandskraft des Zugverbands, jeweils im Massenband- bzw. Massenpunktmodell.
\end{enumerate}

\end{aufgabe}
\vspace{.5cm}
\begin{aufgabe}[Kuppelsto{\ss}/Crash]
Ein dreiteiliger Metro-Triebzug ($m_{w} = 50 \, \mathrm{t}$) soll mit einer automatischen Mittelpufferkupplung ausgestattet werden, die Kuppeln mit $v = 4 \frac{\mathrm{km}}{{h}}$ zul\"asst. Der maximale Hub der Frontkupplung sei auf $s_{max} = 50 \mathrm{mm}$ begrenzt, die Zwischenkupplungen seien starr. Das stehende Fahrzeug ist w\"ahrend des Kuppelns mit der selbstt\"atigen Bremse gebremst.
\begin{enumerate}[a)]
\item Welche Kraft muss \"uber den Verz\"ogerungsweg durchschnittlich herrschen, um die dieses Kuppeln zuzulassen? Hierbei sei die Energie ausschlie{\ss}lich \"uber die Kupplung verzehrt.
\item Was geschieht mit dem stehenden Fahrzeug?
\item Welche Verz\"ogerung herrscht unter den Annahmen von Aufgabe a) im fahrenden Fahrzeug?
\item Bei einem Crash mit einem baugeichen Fahrzeug 
%mit $v = 18 \frac{\mathrm{km}}{{h}}$ 
gem\"a{\ss} EN15227 stehen Energieverzehrelemente mit einem Hub von $s = 200 \mathrm{mm}$ zur Verf\"ugung. Welche Verz\"ogerung und welche Kraft stellt sich ein?
\end{enumerate}
\end{aufgabe}
\end{document}