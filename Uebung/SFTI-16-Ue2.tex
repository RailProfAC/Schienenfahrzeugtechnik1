\documentclass[11pt,a4paper,headsepline]{scrartcl}
\usepackage[utf8]{inputenc}
\usepackage[german]{babel}
\usepackage{amsmath}
\usepackage{amsfonts}
\usepackage{amssymb}
\usepackage{tikz}
\usepackage[left=2cm,right=2cm,top=2cm,bottom=2cm]{geometry}
\usepackage[utf8]{inputenc}
\usepackage{graphicx}
\usepackage{pgfplots}
\usepackage{fancyhdr}
\usepackage[scaled]{helvet}

\usetikzlibrary{calc}
\newtheorem{aufgabe}{Aufgabe}


\renewcommand*\familydefault{\sfdefault}
\renewcommand{\arraystretch}{1.1}

\title{\"Ubung L\"angsdynamik}
\date{}
\makeatletter
\let\Title\@title
\let\Author\@author
\makeatother
\pagestyle{fancy}
\fancyhead[L]{Prof. Dr. Raphael Pfaff}
\fancyhead[R]{84111: Schienenfahrzeugtechnik}
\fancyfoot[L]{Datei: \jobname}
\fancyfoot[R]{Datum: \today
\begin{picture}(0,0)(0,0)\put(.5,0){\includegraphics[height=5cm]{fh_logo}}\end{picture}}

\begin{document}
\maketitle
\thispagestyle{fancy}
\pagestyle{fancy}
\vspace{-2cm}

\begin{aufgabe}[Railway Challenge]
\label{Ta:Acceleration}
Eine Parkbahn-Lokomotive soll gem\"a{\ss} Lastenheft ausgelegt werden. Bestimmen Sie:
\begin{itemize}
	\item Die f\"ur die Acceleration Challenge optimale Masse innerhalb der Grenzen der Spezifikation f\"ur einen konstanten Reibwert $\mu = 0{,}15$.
	\item Den Energiebedarf f\"ur den Dauerbetrieb gem\"a{\ss} Spezifikation.
	\begin{itemize}
		\item Rechnen Sie in eine Masse des Energiespeichers um f\"ur folgende Energiedichten: \vspace{.2cm} \\ 
		\centering
		\begin{tabular}{|c|c|}
		\hline
		Wasserstoff & $1{,}19 \frac{\mathrm{MJ}}{\mathrm{kg}}$ \\ \hline
		Bleiakkumulator & $0{,}11 \frac{\mathrm{MJ}}{\mathrm{kg}}$ \\ \hline
		Mi-Mh-Akku & $0{,}28 \frac{\mathrm{MJ}}{\mathrm{kg}}$ \\ \hline
		Methan & $50 \frac{\mathrm{MJ}}{\mathrm{kg}}$ \\ \hline
		Diesel & $43 \frac{\mathrm{MJ}}{\mathrm{kg}}$ \\ \hline
		\end{tabular}
		\end{itemize}
		\item Bestimmen Sie die Energie der vorgeschriebenen rekuperativen Bremsung und die daraus zu erreichende Geschwindigkeit abh\"angig vom Wirkungsgrad. Weiterhin bestimmen Sie die zu erreichende Fahrstrecke unter der Annahme von einem Fahrwiderstand von $1\%$ der Gewichtskraft abh\"angig vom Wirkungsgrad.
\end{itemize}
\end{aufgabe}
\end{document}